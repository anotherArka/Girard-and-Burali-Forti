\documentclass[10pt]{beamer}
\usepackage{listings}
\usetheme{Boadilla}
\usecolortheme{beaver}

\usepackage[T1]{fontenc}
\usepackage[latin9]{inputenc}
\usepackage{graphicx}
\usepackage{array}
%\usepackage{enumitem}
\usepackage{amsthm,amsmath,latexsym,amssymb,amsmath}
\usepackage{mathrsfs,graphicx,xargs,etoolbox}

\title{Understanding Universes and Univalence}
\author{Arka Ghosh}
\institute{Guided by - Prof. }
\date{}

%commands
\newcommand{\U}{\mathscr{U}}
\newcommand{\R}{\mathbb{R}}

\theoremstyle{definition}
\newtheorem{question}{Question}[section]

\theoremstyle{definition}
\newtheorem{answer}{Answer}[section]


\setbeamercovered{invisible}

\begin{document}

\begin{frame}\label{titlepage}
\titlepage
\end{frame}

%%%%%%%%%%%%%%%%%%%%%%%%%%%%%%%%%%%%%%%%%%%%%%%%%%%%%%%%%%%%%%%%%%%%

\begin{frame}\label{intro}

\visible<1->{
\begin{block}{}
A \emph{formal deduction system} consists of

\begin{itemize}
\item rules of forming valid expressions, and
\item rules of deduction. 
\end{itemize}

\end{block}
}

\visible<2->{
\begin{block}{}
\begin{itemize}
\visible<2->{
\item The usual formal deduction system for mathematics is first-order logic (fol). 
}\visible<3->{
\item A theory such as ZFC is a formal deduction system together with axioms.
}\visible<4->{
\item Type theory is a richer formal deduction system than fol where axioms are replaced
      with rules.
\end{itemize}
}
\end{block}
}

\end{frame}

%%%%%%%%%%%%%%%%%%%%%%%%%%%%%%%%%%%%%%%%%%%%%%%%%%%%%%%%%%%%%%%%%%%%%%%%%%%%%%%%%%%%%%%

\begin{frame}\label{type theory}

\visible<1->{
\begin{block}{}
Basic objects of type theory are terms/elements and types.
\end{block}
}

\visible<2->{
\begin{block}{}
Type theory has two basic forms of judgment/statements:
\begin{itemize}
\item $a : A$ ($a$ is a term of type $A$).
\item $a \equiv b : A$ ($a$ and $b$ are definitionally equal terms in type $A$).
\end{itemize}
\end{block}
}

\visible<3->{
\begin{block}{Some Predefined Types}
\begin{itemize}
\item The \emph{empty type} $\mathbf{0}$.
\item The \emph{unit type} $\mathbf{1}$ with $\star : \mathbf{1}$.
\item The \emph{boolean type} $\mathbf{2}$ with $0_{\mathbf{2}}, 1_{\mathbf{2}} : \mathbf{2} $.
\end{itemize}

\end{block}
}

\visible<4->{
\begin{block}{Some Type Forming Rules}
Given types $A$ and $B$ we can form:
\begin{itemize}
\item The \emph{function type} $A \to B$.
\item The \emph{product type} $A \times B$.
\item The \emph{co-product type} $A + B$.
\end{itemize}
\end{block}
}

\end{frame}

%%%%%%%%%%%%%%%%%%%%%%%%%%%%%%%%%%%%%%%%%%%%%%%%%%%%%%%%%%%%%%%%%%%%

\begin{frame}{Dependent Types}\label{dependent types}

\visible<1->{
\begin{block}{}
In dependent type theory
\begin{itemize}
\item Types can depend on terms.
\item Types can be terms themselves.
\end{itemize}
\end{block}
}

\visible<2->{
\begin{block}{Universes and Type Family}
As types are terms we need types of types. These special types
whose terms are types are called \emph{universes}.\smallskip

Given a type $A$ and universe $\U$ a function $f : A \to \U$ is called a \emph{type family}.
\end{block}
}

\visible<3->{
\begin{block}{Dependent Types}
Given a type family $P : A \to \U $ we can form:
\begin{itemize}
\item The \emph{dependent function type} $\prod_{a : A} P(a)$.
\item The \emph{depedent pair type} $\sum_{a : A} P(a)$.
\end{itemize}

\end{block}
}

\end{frame}

%%%%%%%%%%%%%%%%%%%%%%%%%%%%%%%%%%%%%%%%%%%%%%%%%%%%%%%%%%%%%%%%%%%%

\begin{frame}{An example to give the intuition}\label{example}

\begin{itemize}

\item Consider $S^2 \subseteq \R^3$ as a type, i.e., the terms of this type are points 
      in $S^2$.
      
\item We associate to $p \in S^2$ the type whose terms are tangent vectors at $p$. Then this
      is a type family.
      
\item $\prod$-type corresponds to vector fields (sections).

\item $\sum$-type corresponds to the union of the tangent spaces (i.e. total space).           

\end{itemize}

\end{frame}

%%%%%%%%%%%%%%%%%%%%%%%%%%%%%%%%%%%%%%%%%%%%%%%%%%%%%%%%%%%%%%%%%%%%

\begin{frame}{Proposition as Types}\label{prop as types}

\begin{block}{}
A proposition can be considered as a type $A$ with 
the witnesses/proofs as its terms. So to prove a proposition $A$ we need to show that 
there is a term $a : A$.
\end{block}

\begin{block}

\begin{table}
\caption{Correspondence between logical and type theoritic operations}
\begin{tabular}{ l | l }

\hline
English & Type Theory \\ 
\hline
 
True & $\mathbf{1}$ \\

False & $\mathbf{0}$ \\

$A$ and $B$ & $A \times B$ \\

$A$ or $B$ & $A + B$ \\

$A$ implies $B$ & $A \to B$ \\

Not $A$ & $A \to \mathbf{0}$

\end{tabular}

\end{table}


\end{block}

\end{frame}


%%%%%%%%%%%%%%%%%%%%%%%%%%%%%%%%%%%%%%%%%%%%%%%%%%%%%%%%%%%%%%%%%%%%

\begin{frame}{Universes and Paradoxes in Set Theory}\label{uni and para}

\visible<1->{
\begin{block}{}
The existence of an \emph{universe} in set theory, i.e., \emph{set of all sets} leads to
classical paradoxes due to Cantor, Russell, Burali-Forti. 
\end{block}
}

\visible<2->{
\begin{block}{Burali-Forti Paradox}

\begin{itemize}
\item Consider \emph{ordinals} to be sets with an irreflexive well-ordering (like $<$).

\item Then existence of an universal set implies existence of the collection of all 
      ordinals equal upto order-preserving bijections. Call this collection $Ord$.

\item Order preserving strict inclusions induces an irreflexive well-order $<_o$ on $Ord$.

\item Then $(Ord,<_o) \in Ord$ and $A <_o Ord$ for all ordinals $A$. This leads to a        
      contradiction as $<_o$ is irreflexive.      
\end{itemize}

\end{block}
}

\end{frame}

%%%%%%%%%%%%%%%%%%%%%%%%%%%%%%%%%%%%%%%%%%%%%%%%%%%%%%%%%%%%%%%%%%%

\begin{frame}

\begin{block}{}
This proof can be formulated to type theory to arrive at a contradiction starting from the
rule $\U : \U$ (or even for a finite number of universes). 
Difficulties arise in formulation because
\begin{itemize}
\item unlike sets types have a higher structure,
\item we do not have excluded middle.
\end{itemize}
\end{block}


\begin{block}{}
So we have an infinite sequence of universes 

\[ \U_0 : \U_1 : \U_2 ...\]
\end{block}

\end{frame}

%%%%%%%%%%%%%%%%%%%%%%%%%%%%%%%%%%%%%%%%%%%%%%%%%%%%%%%%%%%%%%%%%%%%

\begin{frame}{Vector Bundles}\label{vec bun}

\visible<1->{
\begin{definition} 
A vector bundle of dimension $k$ over $F$ ($\mathbb{R}$ or $\mathbb{C}$) is a 
bundle $p : E \to B$ where inverse image of each $b \in B$ has a $k$-dimensional
vector space structure over $F$ and the bundle is locally trivial.
\end{definition}
}

\visible<2->{
\begin{example}
Two examples of vector bundles over $S^1$ are the line bundles corresponding to 
the cylinder and m\"obius strip. 
\end{example}
}

\visible<3->{
\begin{question}

\begin{itemize}
\item Are these the only line bundles over $S^1$? \\
\item More generally, can we classify $k$-dimensional vector bundles over a space?
\end{itemize}

\end{question}
}

\end{frame}

%%%%%%%%%%%%%%%%%%%%%%%%%%%%%%%%%%%%%%%%%%%%%%%%%%%%%%%%%%%%%%%%%%%%%

\begin{frame}{Universal Bundles}\label{uni bun}

\begin{answer}
For each $k$ there is a classifying space $B_k$ for vector bundles. 
That is, given a space $X$ there is a bijective correspondence between
vector bundles over $X$ and homotopy classes of maps $X \to B_k$.
\end{answer}

\begin{example}
For line bundles the classifying space is $\R P^{\infty}$.
\end{example}

\end{frame}

%%%%%%%%%%%%%%%%%%%%%%%%%%%%%%%%%%%%%%%%%%%%%%%%%%%%%%%%%%%%%%%%%%%%%

\begin{frame}{Homotopy Type Theory}\label{hott}

\visible<1->{
\begin{block}{}

\begin{itemize}
\item In homotopy type theory (hott) types are interpreted as spaces.
      This way each statement in hott gets a topological interpretation.
\item $A\to B$ is interpreted as the space of continuous functions.
\item $A \times B$ is interpreted as the product space. 
\end{itemize}
      
\end{block}
}

\end{frame}

%%%%%%%%%%%%%%%%%%%%%%%%%%%%%%%%%%%%%%%%%%%%%%%%%%%%%%%%%%%%%%%%%%%%

\begin{frame}

\visible<1->{
\begin{block}{Equality Types}

\begin{itemize}

\item The proposition that two terms of the same type $a, b : A$ are equal 
      is considered as a type $a =_A b$. 
\item It is a so called \emph{indexed inductive type} freely generated by 
      \[{refl} :\prod_{a : A} a =_A a \].
\end{itemize}

\end{block}
}

\visible<2->{ 
\begin{block}{}

\begin{itemize}

\item Topologically $a =_A b$ can be interpreted as the space of paths from $a$ to $b$.
\item Then ${refl}$ can be considered as the subspace of constant paths/loops. 
\item The rule that equality types are generated by reflexivity reflects the fact 
      that the path-space deformation retracts to the subspace of constant loops.

\end{itemize}

\end{block}
}
\end{frame}

%%%%%%%%%%%%%%%%%%%%%%%%%%%%%%%%%%%%%%%%%%%%%%%%%%%%%%%%%%%%%%%%%%%%%

\begin{frame}{Type Families are Fibrations}

\begin{block}{}

\begin{itemize}
\visible<1->{
\item Any type family $P : A \to \U$ can be considered as a fibration 
      ${ {pr}_1 : \sum_{a : A} P(a) \to A }$, where ${pr}_1$ is projection onto the first
      co-ordinate.
}\visible<2->{      
\item This fibration satisfies \emph{homotopy lifting property} automatically.
}\visible<3->{ 
\item In this interpretation functions $f : \prod_{a : A} P(a)$ become sections.
}\visible<4->{
\item A path $p : a =_A b$ induces a fibre transport $p_* : P(a) \to P(b)$. 
     We can even prove that the fibres are equivalent (with a proper notion of equivalence).
}\visible<5->{
\item So we automatically have \textbf{equivalences arising from equalities.}
}
\end{itemize}
\end{block}

\end{frame}

\begin{frame}{Univalence Axiom}



\begin{block}{}
\begin{itemize}
\visible<1->{
\item We can define equivalence ($\simeq$) as functions with \emph{inverses}.
}\visible<2->{
\item Considering types as points in the universe equality gives an equivalence. That is
      we have $(A =_{\U} B) \to (A\simeq B)$.
}\visible<3->{
\item Univalence axiom due to Voedodsky says that this itself is an equivalence. That is,
      $(A =_{\U} B) \simeq (A\simeq B)$.      
}      
\end{itemize}
\end{block}

\end{frame}

%%%%%%%%%%%%%%%%%%%%%%%%%%%%%%%%%%%%%%%%%%%%%%%%%%%%%%%%%%%%%%%%%%%%%

\begin{frame}

\begin{block}{Recall}
Due to Burali-Forti paradox in type theory there can not be a single \emph{universe}. So
we have a infinte sequence of univereses 

\[ \U_0 : \U_1 : \U_2 ...\]
\end{block}

\begin{block}{}
But each of this universes are \emph{univalent}. That is, in these universes 
\textbf{isomorphic structures are equal.}
\end{block}




\end{frame}

%%%%%%%%%%%%%%%%%%%%%%%%%%%%%%%%%%%%%%%%%%%%%%%%%%%%%%%%%%%%%%%%%%%%%

\begin{frame}{References}

\begin{thebibliography}{99}

\bibitem{Hott}
Homotopy Type Theory: Univalent Foundations of Mathematics.

\bibitem{Manin}
A Course in Mathematical Logic for Mathematicians: Yu. I. Manin. 

\bibitem{Husemoller}
Fibre Bundles: Dale Husemoller.


\end{thebibliography}
\end{frame}

\end{document}