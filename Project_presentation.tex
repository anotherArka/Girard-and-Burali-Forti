\documentclass[10pt]{beamer}
\usetheme{Boadilla}

\usepackage[T1]{fontenc}
\usepackage[latin9]{inputenc}
\usepackage{graphicx}
\usepackage{array}
%\usepackage{enumitem}
\usepackage{amsthm,amsmath,latexsym,amssymb,amsmath}
\usepackage{mathrsfs,graphicx,xargs,etoolbox}

\title{Understanding Universes and Univalence}
\author{Arka Ghosh}
\institute{Guided by - Prof. }
\date{}

%commands
\newcommand{\U}{\mathscr{U}}
\newcommand{\R}{\mathbb{R}}

\begin{document}

\begin{frame}\label{titlepage}
\titlepage
\end{frame}

%%%%%%%%%%%%%%%%%%%%%%%%%%%%%%%%%%%%%%%%%%%%%%%%%%%%%%%%%%%%%%%%%%%%

\begin{frame}\label{type theory}

\only<1>{
\begin{block}{}
Type theory is a \emph{language} with a \emph{deductive system}. That is, it has

\begin{itemize}
\item rules of forming valid expressions, and
\item rules of deduction. 
\end{itemize} 
It gives an alternate to first-order logic with ZFC set theory as a foundation of mathematics.
But unlike first-order logic with ZFC it has only \emph{one layer} (we will see what that
means shortly). 
\end{block}
}

\only<2->{
\begin{block}{}
Type theory has two basic forms of judgment/statements:
\begin{itemize}
\item $a : A$ ($a$ is an term/object/element of type $A$).
\item $a \equiv b : A$ ($a$ and $b$ are definionally equal objects in type $A$).
\end{itemize}
\end{block}
}

\only<3->{
\begin{block}{Some Predefined Types}
\begin{itemize}
\item The \emph{empty type} $\mathbf{0}$.
\item The \emph{unit type} $\mathbf{1}$ with $\star : \mathbf{1}$.
\item The \emph{boolean type} $\mathbf{2}$ with $0_{\mathbf{2}}, 1_{\mathbf{2}} : \mathbf{2} $.
\end{itemize}

\end{block}
}

\only<4->{
\begin{block}{Some Type Forming Rules}
Given types $A$ and $B$ we can form:
\begin{itemize}
\item The \emph{function type} $A \to B$.
\item The \emph{product type} $A \times B$.
\item The \emph{co-product type} $A + B$.
\end{itemize}
\end{block}
}

\end{frame}

%%%%%%%%%%%%%%%%%%%%%%%%%%%%%%%%%%%%%%%%%%%%%%%%%%%%%%%%%%%%%%%%%%%%

\begin{frame}{Dependent Types}\label{dependent types}

\only<1->{
\begin{block}{}
We can also have types depending on terms and consider types as terms.
\end{block}
}

\only<2->{
\begin{block}{Universes and Type Family}
To consider types as terms we need types of types. These special types
whose objects are types are called \emph{universes}.\smallskip

Given a type $A$ and universe $\U$ a function $f : A \to \U$ is called a \emph{type family}.
\end{block}
}

\only<3->{
\begin{block}{Dependent Types}
Given a type family $P : A \to \U $ we can form:
\begin{itemize}
\item The \emph{dependent function type} $\prod_{a : A} P(a)$.
\item The \emph{depedent product type} $\sum_{a : A} P(a)$.
\end{itemize}

\end{block}
}
\only<4->{
\begin{block}{An Example}
Ask how to write
\end{block}
}
\end{frame}

%%%%%%%%%%%%%%%%%%%%%%%%%%%%%%%%%%%%%%%%%%%%%%%%%%%%%%%%%%%%%%%%%%%%

\begin{frame}{Universes and Paradoxes in Set Theory}\label{uni and para}

\only<1->{
\begin{block}{}
The existence of an \emph{universe} in set theory, i.e., \emph{set of all sets} leads to
classical paradoxes due to Cantor, Russell, Burali-Forti. 
\end{block}
}

\only<2->{
\begin{block}{Burali-Forti Paradox}

\begin{itemize}
\item Consider \emph{ordinals} to be sets with an irreflexive well-ordering.

\item Then existence of an universal set implies existence of the collection of all 
      ordinals equal upto order-preserving bijections. Call this collection $Ord$.

\item Order preserving strict inclusions induces an irreflexive well-order $<$ on $Ord$.

\item Then $(Ord,<) \in Ord$ and $A < Ord$ for all ordinals $A$. This leads to a contradiction
      as $<$ is irreflexive.      
\end{itemize}

\end{block}
}

\end{frame}

%%%%%%%%%%%%%%%%%%%%%%%%%%%%%%%%%%%%%%%%%%%%%%%%%%%%%%%%%%%%%%%%%%%%

\begin{frame}{Vector Bundles}\label{vec bun}

\only<1->{
\begin{definition} 
A vector bundle of dimension $k$ over $F$ ($\mathbb{R}$ or $\mathbb{C}$) is a 
bundle $p : E \to B$ where inverse image of each $b \in B$ has a $k$-dimensional
vector space structure over $F$ and the bundle is locally trivial.
\end{definition}
}

\only<2->{
\begin{example}
Two examples of vector bundles over $S^1$ are the line bundles corresponding to 
the cylinder and m\"obius strip. 
\end{example}
}

\only<3->{
\begin{example}
A slightly more non-trivial example is the canonical line bundle $\gamma^n_1$
on $\R P^{n-1}$. It is a 
\emph{subbundle} of the product bundle $(\R P^{n-1} \times \R ^n, p, \R ^n)$. Where the 
total space $E$ consists of pairs $([v],x) \in (\R P^{n-1} \times \R ^n)$ such that 
$x$ is in the line corresponding to $[v]$.\smallskip

This definition can be carried on to $n = \infty$ by inductive/weak topology.	
\end{example}
}

\end{frame}

%%%%%%%%%%%%%%%%%%%%%%%%%%%%%%%%%%%%%%%%%%%%%%%%%%%%%%%%%%%%%%%%%%%%%

\begin{frame}{Universal Bundles}\label{uni bun}

\only<1>{
\begin{block}{Canonical Vector Bundle over the \emph{Grassmanian}}
The canonical $k$-dimensional vector bundle over the $k$-dimensional \emph{Grassmanian}
$\gamma^n_k$ is a generalisation of $\gamma^n_1$.\smallskip

$\gamma^{\infty}_k$ is a \emph{classifying space} due to following theorems.
\end{block}
}

\only<2>{
\begin{theorem}[1]\label{T:Ind}
Every vector bundle $\mathscr{E}^k$ over a paracompact space $B$ is $B$-isomorphic
to the induced bundle $f^*(\gamma^{\infty}_k)$ for some ${f : B \to G_k (F^{ \infty })}$.
\end{theorem}

\begin{block}{}
This theorem says that every vector bundle can be written as an induced bundle of 
$\gamma^{\infty}_k$.
\end{block}

}

\only<3>{
\begin{theorem}[2]\label{T:Hom to Eq}
Let $f, g: B \to B'$ be two homotopic maps, where $B$ is a para-compact space, 
and let $\mathscr{E}$ be a vector bundle over $B'$. Then $f^*(\mathscr{E})$ and 
$g^*(\mathscr{E})$ are $B$-isomorphic.
\end{theorem}

\begin{block}{}
This theorem says that for any vector bundle homotopies between maps into it induce 
isomorphism between the induced bundles.
\end{block}

}

\only<4>{
\begin{theorem}[3]\label{T:Eq to Hom}
Let $f, g: B \to G_k(F^n)$ be s.t. $f^*(\gamma_k^n)$ and $g^*(\gamma_k^n)$
are $B$-isomorphic and let ${ j: G_k(F^n) \to G_k(F^{2n}) }$ be the natural inclusion. Then the
maps $jf$ and $jg$ are homotopic for $1 \leq n \leq \infty$. 
\end{theorem}

\begin{block}{}
This theorem says that isomorphisms between induced bundles of $\gamma^{\infty}_k$ 
induce homotopies between the corresponding maps into $\gamma^{\infty}$.
\end{block}
}

\only<5>{

\begin{block}{}
So $\gamma^{\infty}_k$ is a classifying space for $k$-dimensional vector bundles 
in the following sense.
\begin{itemize}
\item Every $k$-dimensional vector bundle can be induced from it.
\item Homotopies between maps into it and isomorphisms between the corresponding induced
      bundles are in one-to-one correspondence. 
\end{itemize}
\end{block}

}

\end{frame}

%%%%%%%%%%%%%%%%%%%%%%%%%%%%%%%%%%%%%%%%%%%%%%%%%%%%%%%%%%%%%%%%%%%%%

\begin{frame}

\begin{thebibliography}{99}

\bibitem{Hott}
Homotopy Type Theory: Univalent Foundations of Mathematics.

\bibitem{Manin}
A Course in Mathematical Logic for Mathematicians: Yu. I. Manin. 

\bibitem{Husemoller}
Fibre Bundles: Dale Husemoller.


\end{thebibliography}
\end{frame}

\end{document}