\documentclass[11pt]{article}

\usepackage[T1]{fontenc}
\usepackage[latin9]{inputenc}
\usepackage{graphicx}
\usepackage{array}
\usepackage{enumitem}
\usepackage{amsthm,amsmath,latexsym,amssymb,amsmath}
\usepackage{mathrsfs,graphicx,xargs,etoolbox}
\usepackage[bookmarks=true]{hyperref}
\usepackage{bookmark}

\setlength{\textwidth}{17cm}
\hoffset=-2.3cm

\setlength{\textheight}{22cm}
\voffset=-1.5cm

\title{\bf{Project Report}}
\author{\textsc{Arka Ghosh}}

\theoremstyle{definition}
\newtheorem{definition}{Definition}[section]

\theoremstyle{plain}
\newtheorem{theorem}[definition]{Theorem}
\newtheorem{proposition}[definition]{Proposition}
\newtheorem{example}[definition]{Example}
\newtheorem{lemma}[definition]{Lemma}
\newtheorem{corollary}[definition]{Corollary}
\newtheorem{Rule}[definition]{Rule}

\theoremstyle{remark}
\newtheorem{remark}[definition]{Remark}

%symbols
\newcommand{\U}{\mathscr{U}}
\newcommand{\Po}{\mathscr{P}}

\begin{document}
\begin{titlepage}

\maketitle

\tableofcontents
\end{titlepage}

\section{Introduction}\label{Intro}
\textbf{\textit{have to write abstract}}\\
Type theory is an alternate to Zermelo-Fraenkel set theory, as a foundational system for mathematics. Homotopy type theory is the type theory where the (propositional) equality is modelled after homotopy. The goal of this project is to understand two fundamental notions of homotopy type theory namely, universes and univalence. In classical mathematics, there are two most used notions of universes:

\begin{enumerate}

\item[1] Zero-level universes in set theory, i.e collection of all sets.

\item[2] Classifying objects in category theory or fibre bundle theory, where objects are 
         classified by maps into to classifying object.  

\end{enumerate}

\noindent Classical paradoxes like the ones due to Russell and Burali-Forti arises if one 
considers the first one(i.e the collection of all sets) as a set. Whereas, classifying objects 
do exist in category theory and for fibre bundles. In type theory, one wants to incorporate 
both these ideas. 

\section{Burali-Forti Paradox}\label{S:BFP}

Now we will see an informal description of the Burali-Forti paradox, which demonstrates a
contradiction arising from the existence of a single universe containing `everything'.\smallskip

Let $Ord$ be the set of all ordered pairs of the form $(A, <_A)$, where $A$ is a set and
and $<_A$ is an irreflexive, well ordering on $A$. Also $(A, <_A)$ and $(B, <_B)$ are 
considered equal/isomorphic (written as $(A,<_A) \cong (B,<_B)$ if there is an order 
preserving bijection between them. I.e. $Ord$ is the universe of all ordinals equal upto 
order preserving bijections \textbf{\textit{give examples}}.\smallskip

Given $a\in A$, let $A_a = \{b \in A,\ b <_A a \}$. $A_a$ inherits an order (say $<_a$)
from $(A, <_A)$. We call $(A_a, <_a)$ be the initial segment of $a$ in $(A, <_A)$. Let
$I_A = \{(A_a, <_a),\ a\in A\}$ be the set of initial segments. Then 
$(I_A,<_I) \cong (A, <_A)$, where $((B,<_B) <_I (C,<_C))$ if there is an order preserving 
strict inclusion (i.e not surjective) from $(B,<_B)$ into $(C,<_C)$. Clearly 
$(A_a,<_a) <_I (A_b,<_b)$ iff $a <_A b$.\smallskip

Now define an order $<$ on $Ord$ as ; $(B,<_B) < (A, <_A)$ if $(B, <_B) \cong (A_a, <_a)$ 
for some $a \in A$ (equivalently if there is an order preserving strict inclusion 
from $(B,<_B)$ into $(A,<_A)$). Then $(Ord, <)$ becomes a well-ordered set 
\textbf{(((a proof .....)))}. So $(Ord, <) \in Ord$. Also notice that as this order is defined 
on the equivalence classes on well-ordered sets upto order preserving bijection, it is 
irreflexive.\smallskip

Now take any $(A, <_A)\in Ord$. Notice that $(I_A, <_I)$ is inherited from ${(Ord, <)}$. So
$(A, <_A) < (Ord, <)$, as $(A,<_A) \cong (I_A, <_I)$.\smallskip

Let $(\mathcal{O}, <')$ be the ordinal where, $\mathcal{O} = Ord\sqcup \{\star\}$, for 
$a,b\in Ord$ $a<'b$ if $a<b$, and $a<\star$ for all $a\in Ord$. Then 
$(Ord,<)< (\mathcal{O},<')$. But $(\mathcal{O},<')<(Ord,<)$ as proved in the last paragraph.
So this gives a contradiction as $<$ is transitive but irreflexive. 

\section{Formulating the Paradox in Type Theory}\label{S:Form in TT}

Now we will see an informal formulation of the Burali-Forti paradox in type theory. 
Difficulties arise in formulating the paradox in type theory because:
\begin{enumerate}
\item Unlike sets types have a higher structure.
\item Under the proposition as types interpretation, proofs of a proposition are not unique.
\end{enumerate}

For example if we define the type of semigroups as:

\[ \text{Semigroup}:\equiv 	\sum_{A : \U} \sum_{m : A \to A \to A} 
    \prod_{x,y,z:A} m(x,m(y,z)) = m(m(x,y),z) \]
    
i.e a type $A$, along with a binary operation $m$, and a proof of associativity of $m$; then
by a semigroup (under the topological interpretation) we mean a space $A$ along with a
continuous function $m : A\times A \to A$ and a continuous family of paths from 
$m(x,m(y,z)) = m(m(x,y),z)$. Hence clearly proof of a proposition need not be unique. So
we will define the property of being a \textbf{\textit{mere proposition}} as:
${ \text{isProp}(A) :\equiv \prod_{x,y : A} (x = y) }$. This essentially captures the idea
that proposition are either true(i.e. a proof exists, or the type is inhabited) or false(i.e
the type is uninhabited). We define the property of being a set as: 
${ \text{isSet} :\equiv \prod_{x,y:A} \text{isProp}(x=y)  }$. This captures the idea that sets
have no higher structure, which is all we need to formulate the paradox and also most of
modern mathematics which uses set theory as a model.   

\begin{thebibliography}{99}

\bibitem{Hott}
Homotopy Type Theory: Univalent Foundations of Mathematics

\bibitem{Manin}
A Course in Mathematical Logic for Mathematicians: Yu. I. Manin 

\end{thebibliography}

\end{document}