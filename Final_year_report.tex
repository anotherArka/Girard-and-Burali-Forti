\documentclass[10pt]{article}

\usepackage[T1]{fontenc}
\usepackage[latin9]{inputenc}
\usepackage{graphicx}
\usepackage{array}
\usepackage{enumitem}
\usepackage{amsthm,amsmath,latexsym,amssymb,amsmath}
\usepackage{mathrsfs,graphicx,xargs,etoolbox}
\usepackage[bookmarks=true]{hyperref}
\usepackage{bookmark}

\setlength{\textwidth}{18cm}
\hoffset=-2.8cm

\setlength{\textheight}{23cm}
\voffset=-2.0cm

\title{\bf{Project Report}}
\author{\textsc{Arka Ghosh}}

\theoremstyle{definition}
\newtheorem{definition}{Definition}[section]

\theoremstyle{plain}
\newtheorem{theorem}[definition]{Theorem}
\newtheorem{proposition}[definition]{Proposition}
\newtheorem{example}[definition]{Example}
\newtheorem{lemma}[definition]{Lemma}
\newtheorem{corollary}[definition]{Corollary}
\newtheorem{Rule}[definition]{Rule}

\theoremstyle{remark}
\newtheorem{remark}[definition]{Remark}

%symbols
\newcommand{\U}{\mathscr{U}}
\newcommand{\Po}{\mathscr{P}}

\begin{document}
\begin{titlepage}

\maketitle

\tableofcontents
\end{titlepage}

\section{Introduction}\label{Intro}
\textbf{\textit{have to write abstract}}\\
Type theory is an alternate to Zermelo-Fraenkel set theory, as a foundational system for mathematics. Homotopy type theory is the type theory where the (propositional) equality is modelled after homotopy. The goal of this project is to understand two fundamental notions of homotopy type theory namely, universes and univalence. In classical mathematics, there are two most used notions of universes:

\begin{enumerate}

\item[1] Zero-level universes in set theory, i.e collection of all sets.

\item[2] Classifying objects in category theory or fibre bundle theory, where objects are 
         classified by maps into to classifying object.  

\end{enumerate}

\noindent Classical paradoxes like the ones due to Russell and Burali-Forti arises if one 
considers the first one(i.e the collection of all sets) as a set. Whereas, classifying objects 
do exist in category theory and for fibre bundles. In type theory, one wants to incorporate 
both these ideas. 

\section{Informal Description of Type Theory}

\subsection{Basic Notions}
Type theory is a deductive system based on two kinds of judgements:

\begin{enumerate}
\item $a : A$, said as ``$a$ is an object of type $A$''.
\item $a \equiv b : A$, said as ``$a$ and $b$ are definitionally equal objects of type $A$''. 
\end{enumerate} 

The idea definitional equality is close to $\alpha$ and $\beta$ reductions in logic. We 
do not distinguish between terms which are equal by definition. For example, in case of 
natural numbers $1 + 0$ and $0 + 1$ are equal by definition of $+$. But the statement that 
for any $n$, $n + 0$ and $0 + n$ is a proposition which needs to be proven. \smallskip

The first-class objects in type theory are types. We have some predefined types such as:
\begin{enumerate}
\item The empty type $\mathbf{0}$.
\item The unit type $\mathbf{1}$ with $\star : \mathbf{1}$.
\item The boolean type $\mathbf{2}$ with $0_{\mathbf{2}}, 1_{\mathbf{2}} : \mathbf{2}$.  
\end{enumerate}
In addition to that we have type forming
rules to define new types from old ones. For example given types $A$ and $B$ we can form: 
the function type $A \to B$, the product type $A \times B$, the co-product type $A + B$ etc.
Unlike set theory, these are first-class objects in type theory. Although their introduction
and elimination rules agree with the usual notions.

\subsection{Dependent Type Theory}
Extra power is added to type theory by letting types depend on terms. For example we can 
define the dependent function type $\prod_{a : A} P(a)$, and the dependent product type
$\sum_{a : A} P(a)$, where $P$ is a function on $A$ s.t. $P(a)$ is a type for $a : A$. 
$f : \prod_{a : A} P(a)$ maps $a : A$ to $f(a) : P(a)$. $(a,b) : \sum_{a : A} P(a)$ where
$b : P(a)$. Notice that to define $P$ we need a type which contains types as their elements.
These special types are called universes. We will discuss them shortly. 
 
\subsection{Proposition as Types}
One brilliance of type theory is that we can make judgements about type theory in type theory
itself. So we do not need a meta theory. A proposition can be considered as a type $A$ with 
the proofs as its elements. So proving a proposition boils down to showing that there is a
term $a : A$. Logical statements like $A \implies B$, $A \wedge B$, $A \vee B$, 
$\forall a. P(a)$, $\exists a. P(a)$ naturally corresponds to type theoritic operations 
$A \to B$, $A \times B$, $A + B$, $\prod_{a : A} P(a)$, $\sum_{a : A} P(a)$, 
respectively.\smallskip

One special kind proposition considered as a type is the equality. I.e. given $a,b : A$ we
write $a =_A b$ to represent the type of witnesses that $a$ is equal to $b$. We say $a$ and
$b$ are propositionally equal when $a =_A b$ is inhabited. As said earlier, given natural
numbers $n$ and $m$, $n+m$ is equal to $m+n$ propositionally, but not definitionally. 

\subsection{Universes and Univalence}

As described 



\section{Burali-Forti Paradox}\label{S:BFP}

Now we will see an informal description of the Burali-Forti paradox, which demonstrates a
contradiction arising from the existence of a single universe containing `everything'.\smallskip

Let $Ord$ be the set of all ordered pairs of the form $(A, <_A)$, where $A$ is a set and
and $<_A$ is an irreflexive, well ordering on $A$. Also $(A, <_A)$ and $(B, <_B)$ are 
considered equal/isomorphic (written as $(A,<_A) \cong (B,<_B)$ if there is an order 
preserving bijection between them. I.e. $Ord$ is the universe of all ordinals equal upto 
order preserving bijections \textbf{\textit{give examples}}.\smallskip

Given $a\in A$, let $A_a = \{b \in A,\ b <_A a \}$. $A_a$ inherits an order (say $<_a$)
from $(A, <_A)$. We call $(A_a, <_a)$ be the initial segment of $a$ in $(A, <_A)$. Let
$I_A = \{(A_a, <_a),\ a\in A\}$ be the set of initial segments. Then 
$(I_A,<_I) \cong (A, <_A)$, where $((B,<_B) <_I (C,<_C))$ if there is an order preserving 
strict inclusion (i.e not surjective) from $(B,<_B)$ into $(C,<_C)$. Clearly 
$(A_a,<_a) <_I (A_b,<_b)$ iff $a <_A b$.\smallskip

Now define an order $<$ on $Ord$ as ; $(B,<_B) < (A, <_A)$ if $(B, <_B) \cong (A_a, <_a)$ 
for some $a \in A$ (equivalently if there is an order preserving strict inclusion 
from $(B,<_B)$ into $(A,<_A)$). Then $(Ord, <)$ becomes a well-ordered set 
\textbf{(((a proof .....)))}. So $(Ord, <) \in Ord$. Also notice that as this order is defined 
on the equivalence classes on well-ordered sets upto order preserving bijection, it is 
irreflexive.\smallskip

Now take any $(A, <_A)\in Ord$. Notice that $(I_A, <_I)$ is inherited from ${(Ord, <)}$. So
$(A, <_A) < (Ord, <)$, as $(A,<_A) \cong (I_A, <_I)$.\smallskip

Let $(\mathcal{O}, <')$ be the ordinal where, $\mathcal{O} = Ord\sqcup \{\star\}$, for 
$a,b\in Ord$ $a<'b$ if $a<b$, and $a<\star$ for all $a\in Ord$. Then 
$(Ord,<)< (\mathcal{O},<')$. But $(\mathcal{O},<')<(Ord,<)$ as proved in the last paragraph.
So this gives a contradiction as $<$ is transitive but irreflexive. 

\section{Formulating the Paradox in Type Theory}\label{S:Form in TT}

Now we will see an informal formulation of the Burali-Forti paradox in type theory. 
Difficulties arise in formulating the paradox in type theory because:
\begin{enumerate}
\item Unlike sets types have a higher structure.
\item Under the proposition as types interpretation, proofs of a proposition are not unique.
\end{enumerate}

For example if we define the type of semigroups as:

\[ \text{Semigroup}:\equiv 	\sum_{A : \U} \sum_{m : A \to A \to A} 
    \prod_{x,y,z:A} m(x,m(y,z)) = m(m(x,y),z) \]
    
i.e a type $A$, along with a binary operation $m$, and a proof of associativity of $m$; then
by a semigroup (under the topological interpretation) we mean a space $A$ along with a
continuous function $m : A\times A \to A$ and a continuous family of paths from 
$m(x,m(y,z)) = m(m(x,y),z)$. Hence clearly proof of a proposition need not be unique. So
we will define the property of being a \textbf{\textit{mere proposition}} as:
${ \text{isProp}(A) :\equiv \prod_{x,y : A} (x = y) }$. This essentially captures the idea
that proposition are either true(i.e. a proof exists, or the type is inhabited) or false(i.e
the type is uninhabited). We define the property of being a \textbf{\textit{set}} as: 
${ \text{isSet} :\equiv \prod_{x,y:A} \text{isProp}(x=y)  }$. This captures the idea that sets
have no higher structure, which is all we need to formulate the paradox and also most of
modern mathematics which uses set theory as a model. Because these are not first-class 
objects in type theory a lot of work in formulating the paradox will be proving that certain
objects are sets or mere propositions.\smallskip

Another source of difficulty is the absence of excluded middle in type theory. To bypass that
we will need the following definition:

\begin{definition}\label{D:Acc}
Given a set $A$, and a binary relation $(-<-) : A \to A \to Prop$. Now define being 
accesible by $<$ (i.e $\text{Acc}_{<} : A \to \U$) as : If $b$ is accesible for every $b<a$ 
then $a$ is accesible. I.e, it has one constructor 
\[ \text{acc}_< : \prod_{a : A} \left( \prod_{b : A} (b<a) \to \text{Acc}(b) \right) 
\to \text{Acc}(a). \]

\end{definition}
 
This a form of transfinite induction in type theory which we use to prove that certain
properties hold for an element in a type assuming they hold for elements less than it.
It can be proven that accessibility is a mere proposition. 

\begin{definition}\label{D:WF}
We will say a binary relation $<$ on $A$ is \textbf{\textit{well-founded}} if every element
is accessible (i.e we have $f : \prod_{a : A} \to Acc(a)$. Well-foundedness is a mere 
proposition as $Acc$ is so. Notice that if $<$ on $A$ is well-founded then for some 
$P : A \to \U$ proving $\prod_{a:A} acc(a) \to P(a)$ is enough to prove $\prod_{a : A} P(a)$.
\end{definition}

\begin{definition}\label{D:Ext}
We will call $<$ \textbf{\textit{extensional}}, if it is well-founded and we have:
\[ \left( \prod_{c : A}  (c<a) \iff (c<b) \right) \to (a = b),\]
where for two mere propositions $A$ and $B$, we write $A\iff B$ to say that there are 
$f : A \to B$ and $g : B \to A$ (using the definition of $\text{isProp}$ it can be proven 
that $f$ is an equivalence). Notice that extensionality is same as saying that elements 
with equal initial segments are equal.
\end{definition} 

Extensionality is a mere proposition as given $a,b:A$, $a<b$ is a mere proposition, and so
is well-foundedness. Using univalence we can prove that the type of extensional well-founded
relations is a set.

\begin{definition}\label{D:sim}
If $(A,<)$ and $(B,<)$ are well-founded and extensional, a \textbf{\textit{simulation}} is
a function $f : A \to B$ is such that:
\begin{enumerate}
\item if $a < a'$, then $f(a) < f(a')$, and
\item for all $a : A$ and $b : B$, if $b < f(a)$, then there merely exists an $a' < a$ with
      $f(a') = b$
\end{enumerate}
\end{definition} 

A simulation is essentially a function from one ordinal(defined next) to another, 
resepecting the order. Using well-founded induction we can prove that each simulation is 
injective. Because simulations are functions from one set to another and $a<b$ is a mere 
proposition, injectivity implies that being a simulation is also mere proposition. Using 
induction on $<$ we can prove that given for well-founded and extensional $(A,<)$ and $(B,<)$, 
there is at most one simulation $f:A \to B$. If we write $A \leq B$ to mean there exists a 
simulation $A \to B$, then $A \leq B$ is a mere proposition. Moreover it is a partial order, 
i.e $A \leq B$ and $B \leq A$ implies $A = B$. 

\begin{definition}\label{D:Ordinal}
An \textbf{\textit{ordinal}} is a set $A$ with an extensional well-founded relation $<$, which
is also transitive. Given an universe $\U_i$ the type of ordinals in $\U$, $\text{Ord}_{\U}$
can be defined as an element in the next universe $\U_{i+1}$.
\end{definition}

Because $a < b$, well-foundedness and extensionality are mere propositions, so are
transitivity and the property of being an ordinal.\smallskip

Given an ordinal $A$ and $a : A$ let $A_a :\equiv \{ b : A\ |\ b < a\}$ denote the 
\textbf{\textit{initial segment}} of $a$ in $A$. Clearly $A_a$ is also an ordinal. If 
$A_a = A_b$ as ordinals, then the isomorphism should respect the order in $A$. Hence by 
extensionality $a = b$. Hence the function $a \to A_a$ is an injection.

\begin{definition}\label{D:Bdd sim}
Given two ordinals $A$ and $B$, a simulation $f : A \to B$ is said to be 
\textbf{\textit{bounded}} if there exists $b : B$ such that $A = B_b$. 
\end{definition}

Note that from the discussion above boundedness is a mere proposition. We will write 
$A < B$ to denote that there exists a bounded simulation from $A$ to $B$. Because being a 
simulation is a mere property, so is $<$. Using accessibility we can prove that any 
well-founded relation is irreflexive. Using well-founded induction we can prove that $<$
is a well-founded extensional relation on $\text{Ord}_{\U}$. Transitivity follows from 
the definition of a simulation. So we have the following theorem:
\begin{theorem}
$(\text{Ord}_{\U_i},<) : \U_{i+1}$ is an ordinal.
\end{theorem}

Given an ordinal $A$ the map $a \mapsto A_a$ is a simulation $A \to \text{Ord}$ as it is 
injective and respects the orders. Using univalence we can prove that $A = \text{Ord}_A$. 
If there is a single universe $\U$ containing everything then 
$(\text{Ord}_{\U},<) : \text{Ord}_{\U}$. So $A < \text{Ord}_{\U}$ for every ordinal 
$A : \text{Ord}_{\U}$. But that contradicts irreflexivity of $<$.

\section{Understanding Univalence}

\begin{thebibliography}{99}

\bibitem{Hott}
Homotopy Type Theory: Univalent Foundations of Mathematics.

\bibitem{Manin}
A Course in Mathematical Logic for Mathematicians: Yu. I. Manin. 

\end{thebibliography}

\end{document}