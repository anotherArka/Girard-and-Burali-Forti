\documentclass[11pt]{article}

\usepackage[T1]{fontenc}
\usepackage[latin9]{inputenc}
\usepackage{graphicx}
\usepackage{array}
\usepackage{enumitem}
\usepackage{amsthm,amsmath,latexsym,amssymb,amsmath}
\usepackage{mathrsfs,graphicx,xargs,etoolbox}
\usepackage[bookmarks=true]{hyperref}
\usepackage{bookmark}

\setlength{\textwidth}{19cm}
\hoffset=-3.3cm

\setlength{\textheight}{24cm}
\voffset=-2.5cm

\title{\bf{Project Report}}
\author{\textsc{Arka Ghosh}}

\theoremstyle{definition}
\newtheorem{definition}{Definition}[section]

\theoremstyle{plain}
\newtheorem{theorem}[definition]{Theorem}
\newtheorem{proposition}[definition]{Proposition}
\newtheorem{example}[definition]{Example}
\newtheorem{lemma}[definition]{Lemma}
\newtheorem{corollary}[definition]{Corollary}
\newtheorem{Rule}[definition]{Rule}

\theoremstyle{remark}
\newtheorem{remark}[definition]{Remark}

%symbols
\newcommand{\U}{\mathscr{U}}
\newcommand{\Po}{\mathscr{P}}

\begin{document}
\begin{titlepage}

\maketitle

\tableofcontents
\end{titlepage}

\section{Introduction}\label{Intro}
\textbf{\textit{have to write abstract}}\\
Type theory is an alternate to Zermelo-Fraenkel set theory, as a foundational system for mathematics. Homotopy type theory is the type theory where the (propositional) equality is modelled after homotopy. The goal of this project is to understand two fundamental notions of homotopy type theory namely, universes and univalence. In classical mathematics, there are two most used notions of universes:

\begin{enumerate}

\item[1] Zero-level universes in set theory, i.e collection of all sets.

\item[2] Classifying objects in category theory or fibre bundle theory, where objects are 
         classified by maps into to classifying object.  

\end{enumerate}

\noindent Classical paradoxes like the ones due to Russell and Burali-Forti arises if one 
considers the first one(i.e the collection of all sets) as a set. Whereas, classifying objects 
do exist in category theory and for fibre bundles. In type theory, one wants to incorporate 
both these ideas. 

\section{Burali-Forti Paradox}\label{S:BFP}

Now we will see an informal description of the Burali-Forti paradox, which demonstrates a
contradiction arising from the existence of a single universe containing `everything'.\smallskip

Let $Ord$ be the set of all ordered pairs of the form $(A, <_A)$, where $A$ is a set and
and $<_A$ is an irreflexive, well ordering on $A$. Also $(A, <_A)$ and $(B, <_B)$ are 
considered equal/isomorphic (written as $(A,<_A) \cong (B,<_B)$ if there is an order 
preserving bijection between them. I.e. $Ord$ is the universe of all ordinals equal upto 
order preserving bijections \textbf{\textit{give examples}}.\smallskip

Given $a\in A$, let $A_a = \{b \in A,\ b <_A a \}$. $A_a$ inherits an order (say $<_a$)
from $(A, <_A)$. We call $(A_a, <_a)$ be the initial segment of $a$ in $(A, <_A)$. Let
$I_A = \{(A_a, <_a),\ a\in A\}$ be the set of initial segments. Then 
$(I_A,<_I) \cong (A, <_A)$, where $((B,<_B) <_I (C,<_C))$ if there is an order preserving 
strict inclusion (i.e not surjective) from $(B,<_B)$ into $(C,<_C)$. Clearly 
$(A_a,<_a) <_I (A_b,<_b)$ iff $a <_A b$.\smallskip

Now define an order $<$ on $Ord$ as ; $(B,<_B) < (A, <_A)$ if $(B, <_B) \cong (A_a, <_a)$ 
for some $a \in A$ (equivalently if there is an order preserving strict inclusion 
from $(B,<_B)$ into $(A,<_A)$). Then $(Ord, <)$ becomes a well-ordered set 
\textbf{(((a proof .....)))}. So $(Ord, <) \in Ord$. Also notice that as this order is defined 
on the equivalence classes on well-ordered sets upto order preserving bijection, it is 
irreflexive.\smallskip

Now take any $(A, <_A)\in Ord$. Notice that $(I_A, <_I)$ is inherited from ${(Ord, <)}$. So
$(A, <_A) < (Ord, <)$, as $(A,<_A) \cong (I_A, <_I)$.\smallskip

Let $(\mathcal{O}, <')$ be the ordinal where, $\mathcal{O} = Ord\sqcup \{\star\}$, for 
$a,b\in Ord$ $a<'b$ if $a<b$, and $a<\star$ for all $a\in Ord$. Then 
$(Ord,<)< (\mathcal{O},<')$. But $(\mathcal{O},<')<(Ord,<)$ as proved in the last paragraph.
So this gives a contradiction as $<$ is transitive but irreflexive. 

\section{Formulating the Paradox in Type Theory}\label{S:Form in TT}

Now we will try to formulate the paradox in homotopy type theory. More specifically, 
we will arrive at a contradiction assuming the existence of a single universe $\U$, and
assuming univalence axiom. Our description will be informal, because there are a good number 
of resources available for a formal proof, although not many of them explain the idea 
\textbf{((((need a better sentence and to say that the proof is taken from
the HOTT book))))}.\smallskip

\subsection{Notion of Sets and (Mere) Propositions}
One of the main difficulties in formulating the paradox in type theory is that, unlike sets
types have a higher categorical/ homotopical structure arising from the propsitional 
equality. Same problem arises from the propositions as types interpretation. So sets and 
propositions (more specifically, the property of being a set or a proposition) are defined as
\[ isSet(A) :\equiv \prod_{x,y:A} \prod_{p,q : x = y} (p=q)\] and,
\[ isProp(A) :\equiv \prod_{x,y:A} (x = y).\]

I.e. $A$ is a set when any two proofs of equality is equal to each other. Topologically this 
is equivalent to saying that $A$ has discreet topology. $A$ is a proposition when any two 
elements of $A$ are equal. Toplogically this is equivalent to saying that $A$ is either empty
or homotopy equivalent to a point. Logically this means proof of the proposition is unique. 
During the course of the prooof we will see why these definitions help us. We will use the name 
\textbf{\textit{mere proposition}} to distinguish between the types $A$ for which we have
$isProp(A)$, and types representing propositions in general. \smallskip

From now on we will work with only with the types which are either a set or a mere 
proposition. Although sometimes we will have to prove that a newly defined type is indeed 
so, as it is not automatic. We will use $Set_{\U}$ and $Prop_{\U}$ to denote the type
whose elements are the types $A : \U$ for which we have $isSet(A)$ and $isProp(A)$ 
respectively. We will omit the subscript $\U$ when it will be clear from the context. 

\begin{definition}\label{D:Power set}
Given a set $A$, the power set of $A$(written as $\Po (A)$ is defined as:
\[ \Po (A) :\equiv A \to Prop.\]
Using univalence it can be proved that it is a set.
\end{definition}

\begin{definition}\label{D:Subset}
Given $P : A \to Prop$, we call the type $\sum_{a : A}  P(a)$ a \textbf{\textit{subtype}}
of $A$. Because $P(a)$ is a mere proposition for $a : A$, equality for a subtype is generated by 
the equality between the first co-ordinates. Hence, if $A$ is a set and then any subtype of $A$ 
is also a set. In that case we denote it as ${\{ a : A\ |\ P(a) \}}$, and call it to be a 
\textbf{\textit{subset}} of $A$.
\end{definition}

Given two sets $A$ and $B$ the type $A \to B$ is also becomes a set using univalence. Given
$f : A \to B$ it would be nice if we can define the image of $f$ as a subset of $B$. But
given $b : B$ the type $\sum_{a : A} f(a) = b$ is not a mere proposition. So the 
straighforward way of defining the image as $\sum_{b : B} (\sum_{a : A} f(a) = b)$ does not
work. So we will need an additional concept called propositional resizing.

\begin{definition}\label{D:Prop resize}
Given a type $A$ the \textbf{\textit{propositional resizing}} of $A$, denoted as $|| A ||$, is
a type generated by two constructors:
\begin{enumerate}
\item For $a : A$, we have $|a| : A$,
\item For $x,y : A$, we have $x = y$.
\end{enumerate}
The first constructors says that if $A$ is inhabited then $|| A ||$ is inhabited. The second
one ensures that $|| A ||$ is a mere proposition. The recursion principle of $|| A ||$ is:
If $B$ is a mere proposition and we have $f : A \to B$, then there is an induced 
$g : || A || \to B$ such that $g(|a|) \equiv f(a)$ for all $a : A$. 
\end{definition}
 
Equipped with this concept now we will define the image of a function as a subset of the 
codomain. 

\begin{definition}\label{D:Image}
Given two sets $A$ and $B$, and $f : A \to B$, the image of $f$ is defined as:
\[ im(f) :\equiv \{ b : B \text{ s.t. } || \Sigma_{a : A} f(a) = b|| .\}\]
\end{definition}

\subsection{The Category of Ordinals}\label{SS:Cat of ord}

Having defined sets inside type theory, now we want to define ordinals. Because we do not
have excluded middle in type theory, our definition will be slightly different than the
set theoritic one. Intuitively, ordinals will be defined as sets with an additional structure.
We will also need a definition of functions betweeen ordinals respecting these structures. 
To address these issues we will need the notions of `categories' and `structures' inside type 
theory.

\begin{definition}\label{D:Acc}
Given a set $A$, and a binary relation $(-<-) : A \to A \to Prop$. Now define being 
accesible by $<$ (i.e $Acc_{<} : A \to \U$) as : If $b$ is accesible for every $b<a$ 
then $a$ is accesible. I.e, it has one constructor 
\[ acc_< : \prod_{a : A} \left( \prod_{b : A} (b<a) \to Acc(b) \right) \to Acc(a). \]
\end{definition}

The induction principle for $Acc$ is somewhat complicated. But it is proved in \cite{Hott} 
that accesibility is a mere proposition. So given $P : A \to \U$, to prove 
$\prod_{a : A} Acc(a) \to P(a)$ (i.e. give an inhabitant) , it is enough to define :
\[ f : \prod_{a : A} \left( \prod_{b : A} (b<a) \to Acc(b) \times P(b) \right) \to P(a) .\]

\begin{definition}\label{D:WF}
We will say a binary relation $<$ on $A$ is \textbf{\textit{well-founded}} if every element
is accessible (i.e we have $f : \prod_{a : A} \to Acc(a)$. Well-foundedness is a mere 
proposition as $Acc$ is so. Notice that if $<$ on $A$ is well-founded then for some 
$P : A \to \U$ proving $\prod_{a:A} acc(a) \to P(a)$ is enough to prove $\prod_{a : A} P(a)$.
\end{definition}

The following lemma will be useful later:
\begin{lemma}\label{L:Wf implies irr}
Any well-founded relation is irreflexive.
\end{lemma}

\begin{proof}
We will use induction on $acc_<$. Take $a : A$. Assume $\neg(b < b)$ for all $b<a$ 
(induction hypothesis). Then if $a < a$ we have $\neg(a < a)$ using induction hypothesis,
resulting in a contradiction.
\end{proof}

\begin{definition}\label{D:Ext}
We will call $<$ \textbf{\textit{extensional}}, if it is well-founded and we have:
\[ \left( \prod_{c : A}  (c<a) \iff (c<b) \right) \to (a = b),\]
where for two mere propositions $A$ and $B$, we write $A\iff B$ to say that there are 
$f : A \to B$ and $g : B \to A$ (using the definition of $isProp$ it can be proven that $f$ 
is an equivalence). Notice that extensionality is same as saying that elements with equal 
initial segments are equal.
\end{definition} 

\begin{definition}
An \textbf{\textit{ordinal}} is a set $A$ with a relation $<$ which is well-founded, 
extensional and transitive.
\end{definition}

Notice that the definition of ordinals here is slightly different than the set theory. This
is necessary because we do not assume excluded middle (whose general version is 
inconsistent with univalence. Check corollary 3.2.7 of \cite{Hott} for a proof). Although
the version of LEM for mere propositions:
\[ \text{LEM} :\equiv \prod_{A : \U} \left((isProp(A) \to (A + (\neg A))\right) \]  
is consistent with univalence. And assuming LEM, $<$ is well-founded if and only if every
non-empty subset $B:\Po (A)$ merely has a minimal element (Lemma 10.3.8 of \cite{Hott}).

\begin{definition}\label{D:sim}
If $(A,<)$ and $(B,<)$ are extensional and well-founded, a \textbf{\textit{simulation}} is
a function $f : A \to B$ is such that:
\begin{enumerate}
\item if $a < a'$, then $f(a) < f(a')$, and
\item for all $a : A$ and $b : B$, if $b < f(a)$, then there merely exists an $a' < a$ with
      $f(a') = b$
\end{enumerate}
\end{definition} 

A subset $C : \Po (A)$ is called an initial segment of $A$ if $c \in C$ and $a < c$ implies
$a \in C$. Using a double well-founded induction it can be proven that every simulation is
injective (Lemma 10.3.12 of \cite{Hott}). So $\sum_{a : A} (f(a) = b)$ is a mere proposition.
Hence the image of a simulation $f : (A,<) \to (B,<)$ is an initial segment of $A$. Clearly
the other direction is also true. Hence, by univalence every simulation $A \to B$ is equal
to the inclusion of some initial segment of $B$.

A simulation is basically a function from one ordinal to another resepecting the order. Now
we will defined categories in type theory and prove that there is a category with ordinals
as its objects and simulations as arrows.

\begin{definition}\label{D:PreCat}
A \textbf{\textit{precategory}} $A$ has the following:

\begin{enumerate}
\item A type $A_0$, whose elements are are called \textbf{\textit{objects}}. We write $a : A$
    for $a : A_0$ for simplicity.

\item Given $a, b : A$, a set $hom_A(a,b)$, whose elements are called 
    \textbf{\textit{arrows}} or \textbf{\textit{morphisms}}.

\item For each $a : A$, a morphism $1_a : hom_A (a,a)$ , called the 
    \textbf{\textit{identity morphism}}.
    
\item For each $a, b, c : A$, a function 
    \[hom_A (b,c) \to hom_A (a,b) \to hom_A (a,c),\]
    called \textbf{\textit{composition}}. It is denoted as $g \mapsto f \mapsto g \circ f$.

\item For $a, b : A$ and $f : hom_A (a,b )$ , we have $f = 1_b \circ f$ and 
    $f = f \circ 1_a$.
    
\item For $a, b, c, d : A$ and $f : hom A ( a, b )$ ,$g : hom A ( b, c )$ , 
    $h : hom A ( c, d )$ , we have 
    \[ h \circ (g \circ f) = (h \circ g) \circ f .\]
    

\end{enumerate}

\end{definition} 

The following lemma which can be proved using induction on $<$ says that the category is a 
poset.

\begin{lemma}\label{L:Sim is uniq}
For extensional well-founded $(A,<)$ and $(B,<)$, there is at most one simulation $f : A \to B$.
\end{lemma}

So, given extensional well-founded relations $(A, <_A)$ and $(B, <_B)$ we can write $A \leq B$ 
to denote that there is a simulation from $A$ to $B$ respecting the relations. Using univalence,
and the fact that $\leq$ is reflexive and transitive we can prove that it is also 
antisymmetric.

\begin{definition}\label{D:Ordinal}
An \textbf{\textit{ordinal}} is a set $A$ with an extensional well-founded relation $<$, which
is also transitive.
\end{definition}

For example, the usual order on $\mathbb{N}$ makes it an ordinal. Let $Ord$ denote the type of
ordinals. We already know that it is a set by previous results. Now we will show that it admits
a well-founded relation.\smallskip

Given an ordinal $A$ and $a : A$ let $A_a :\equiv \{ b : A\ |\ b < a\}$ denote the 
\textbf{\textit{initial segment}} of $a$ in $A$. Clearly $A_a$ is also an ordinal. If 
$A_a = A_b$ as ordinals, then the isomorphism should respect the order in $A$. Hence by 
extensionality $a = b$. Hence the function $a \to A_a$ is an injection
\textbf{\textit{((((define it))))}} $A \to Ord$. 

\begin{definition}\label{D:Bdd sim}
Given two ordinals $A$ and $B$, a simulation $f : A \to B$ is said to be 
\textbf{\textit{bounded}} if there exists $b : B$ such that $A = B_b$. Note that from the
discussion above boundedness is a mere proposition. We will write $A < B$ to denote that 
there exists a bounded simulation from $A$ to $B$. Because being a simulation is a 
mere property, so is $<$.
\end{definition}

Now we will prove that $(Ord_{\U_i},<)$ is an ordinal in $\U_{i+1}$, i.e. 
$(Ord_{\U_i},<) : Ord_{\U_{i+1}}$. It is in a higher universe as we need to go up at least 
one step to define $<$ on $Ord_{\U_i}$.

\begin{theorem}\label{T:Ord is ord}
$(Ord,<)$ is an ordinal.
\end{theorem}

\begin{proof}
Let $A,B$ and $C$ be ordinals such that $A<B$ and $B<C$. Let $f : A \to B_b$ and $g : B\to C_c$
are simulations. Then $g \circ f : A \to C_{g(b)}$ is a simulation. So $A < C$. Hence 
$<$ is transitive. \smallskip

Let $A$ be an ordinal, and $a : A$. We will use induction of $acc_<$ along with 
well-foundedness to prove that $A_a$ is accessible. Let $A_b$ be accessible for all $b < a$
(induction hypothesis). Using the definition of accessibility it is enough to prove that 
$B$ is accessible for all $B < A_a$. But $B < A_a$ means $B = (A_a)_b = A_b$, for some 
$b < a$. Hence $B$ is accesible using induction hypothesis. So $A_a$ is accessible for
all $a : A$.\smallskip

Now take any $B < A$. Then $B = A_a$ for some $a : A$ and hence is accessible. Hence $A$ is
accessible. Hence $(Ord,<)$ is well-founded.\smallskip

Now say for some $A,B : Ord)$ we have $\prod_{C : Ord} (C < A) \iff (C < B)$. Then for all
$a : A$ we have $A_a < B$, and hence and unique $b : B$ such that $A_a = B_b$. Define 
$f : A \to B$  using this correspondence. Similarly we can defined $g : B \to A$ and prove
that $f$ and $g$ are inverses of each other. Clearly $f$ and $g$ respect the orders on
$A$ and $B$. So $A \cong B$. Using univalence $A = B$. So $(Ord, <)$ is extensional.
\end{proof}

\begin{lemma}
$<$ on $Ord$ is irreflexive.
\end{lemma}

\begin{proof}
Say $A < A$. Then there is a simulation $f : A \to A_a$ for some $a : A$. But then composing
with the inclusion $A_a \hookrightarrow A$ we get a simulation $g : A \to A$ such that
$g(a) \neq a$. Contradicting the uniqueness of simulations.
\end{proof}

The next lemma will give us the last missing ingredient to prove that the existence of a single
universe is inconsistent.

\begin{lemma}\label{L:succ of ord}
If $A : Ord_{\U}$ then there is a $B : Ord_{\U}$ such that $A < B$. 
\end{lemma}

\begin{proof}
Let $B :\equiv A + \mathbf{1}$. With $\star : \mathbf{1}$ greater than every $a : A$. Then
$A : B_{\star}$ hence $B < A$.
\end{proof}

Now assume that there is a singl universe $\U$ containing everything. Then 
$(Ord_{\U},<) : Ord_{\U}$. Now notice the following:

\begin{lemma}\label{L:A<Ord}
For any $(A, <) : Ord_{\U}$, $A = (Ord_{\U})_A$. I.e. any ordinal is isomorphic to the
set contaning its initial segment equipped with the natural order.
\end{lemma}

\begin{proof}
Define $f : A \to Ord_{\U}$ as $f(a) :\equiv A_a$. Clearly $f$ is a simulation. Define
$g : (Ord_{\U})_A \to A$ as $g(C) = c$ where $c : A$ is such that $C = A_c$. Cleary $f$
(restricted to $(Ord_{\U})_A$) and $g$ are inverses of each other, respecting the 
corresponding orders. Hence using univalence $(A, <) : Ord_{\U}$, $A = (Ord_{\U})_A$.
\end{proof}

Using \ref{L:Wf implies irr}, $<$ is irreflexive. But Using the last lemma we have 
$Ord_{\U} < Ord_{\U}$ contradicting the irreflexivity of $<$.

\begin{thebibliography}{99}

\bibitem{Hott}
Homotopy Type Theory: Univalent Foundations of Mathematics

\bibitem{Manin}
A Course in Mathematical Logic for Mathematicians: Yu. I. Manin 

\end{thebibliography}

\end{document}