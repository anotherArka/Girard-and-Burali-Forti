\documentclass[10pt]{article}

\usepackage[T1]{fontenc}
\usepackage[latin9]{inputenc}
\usepackage{graphicx}
\usepackage{array}
\usepackage{enumitem}
\usepackage{amsthm,amsmath,latexsym,amssymb,amsmath}
\usepackage{mathrsfs,graphicx,xargs,etoolbox}
\usepackage[bookmarks=true]{hyperref}
\usepackage{bookmark}

\setlength{\textwidth}{18cm}
\hoffset=-2.8cm

\setlength{\textheight}{23cm}
\voffset=-2.0cm

\title{\bf{Project Report}}
\author{\textsc{Arka Ghosh}}

\theoremstyle{definition}
\newtheorem{definition}{Definition}[section]

\theoremstyle{plain}
\newtheorem{theorem}[definition]{Theorem}
\newtheorem{proposition}[definition]{Proposition}
\newtheorem{example}[definition]{Example}
\newtheorem{lemma}[definition]{Lemma}
\newtheorem{corollary}[definition]{Corollary}
\newtheorem{Rule}[definition]{Rule}

\theoremstyle{remark}
\newtheorem{remark}[definition]{Remark}

%symbols
\newcommand{\U}{\mathscr{U}}
\newcommand{\Po}{\mathscr{P}}

\begin{document}
\begin{titlepage}

\maketitle

\tableofcontents

\begin{abstract}
The aim of this project is to understand the concept of \emph{universes} 
in homotopy type theory. It is divided into two parts. In the first part, we describe 
a type-theoretic analogue of the Burali-Forti paradox arising from the assumption 
that a single universe contains everything. In the second part, we try to interpret 
universes along with the univalence axiom, as universal bundles. 
\end{abstract}

\end{titlepage}

\section{Introduction}\label{Intro}
Type theory is an alternate to first-order logic with ZFC set theory as a model for a
foundation of mathematics. It was first proposed by Bertrand Russell to avoid paradoxes in 
set theory. After that, it has been reformulated again and again by several mathematicians. 
The one we are going to discuss here is closely related to Per Martin-L\"of's intuitionistic 
type theory, with an additional axiom called univalence (due to Voevodsky).


\section{Informal Description of Type Theory}

\subsection{Basic Notions}

The basic notions in type theory are terms (objects, elements) and types.  We write 
$a : A$ to mean $a$ is a term of type $A$. This is a form of judgment in type theory.
There is another form of judgment in type theory called definitional equality. We write
$a \equiv b : A$ to say that $a$ and $b$ are equal by definition in type $A$. We do not
distinguish between terms which are definitionally equal. We have some predefined types such 
as the \emph{empty type} $\mathbf{0}$, the \emph{unit type} $\mathbf{1}$, 
the \emph{boolean type} $\mathbf{2}$. In addition to that we have \emph{type forming rules} 
to define new types from old ones. For example given types $A$ and $B$ we can form the 
function type $A \to B$, the product type $A \times B$, the co-product type $A + B$ etc. 
Their definitions agree with the usual notions.\smallskip

In homotopy type theory types are interpreted as spaces. So $f : A \to B$ is 
interpreted as a continuous function. $A \times B$ is interpreted as the product space.
This way every statement in type theory gets a topological interpretation.

\subsection{Dependent Type Theory}
Extra power is added to type theory by letting types depend on terms and regarding types as
terms. For example we can define the dependent function type $\prod_{a : A} P(a)$, and the 
dependent product type $\sum_{a : A} P(a)$, where $P$ is a function on $A$ s.t. $P(a)$ is a 
type for $a : A$. $f : \prod_{a : A} P(a)$ maps $a : A$ to 
$f(a) : P(a)$. $(a,b) : \sum_{a : A} P(a)$ where $b : P(a)$. Notice that to define 
$P$ we need a type which contains types as their elements.
These special types are called universes. We will discuss them shortly. 
 
\subsection{Proposition as Types}
One brilliance of type theory is that we can make judgments about type theory in type theory
itself. So we do not need a meta theory. A proposition can be considered as a type $A$ with 
the witnesses/proofs as its elements. So to prove a proposition $A$ we need to show that 
there is a term $a : A$. Logical statements like $A \implies B$, $A \wedge B$, $A \vee B$, 
$\forall a. P(a)$, $\exists a. P(a)$ naturally corresponds to type theoretic operations 
$A \to B$, $A \times B$, $A + B$, $\prod_{a : A} P(a)$, $\sum_{a : A} P(a)$
respectively.\smallskip

\subsection{Equality Types}
One special kind of propositions considered as a type is the equality. That is, 
given $a,b : A$ we write $a =_A b$ to represent the type of witnesses that $a$ is equal to $b$. 
It is a so called \emph{indexed inductive type} freely generated by 
${refl} :\prod_{a : A} a =_A a$. We say $a$ and $b$ are propositionally equal 
when $a =_A b$ is inhabited. For example given natural numbers $n$ and $m$, 
$n+m$ is equal to $m+n$ propositionally but not by definition.
\smallskip

Topologically, $a =_A b$ can be interpreted as the space of paths from $a$ to $b$. Then 
${refl}$ can be considered as the subspace of constant paths. The rule that equality
types are generated by reflexivity reflects the fact that the path-space deformation retracts
to the subspace of constant loops.

\subsection{Universes and Univalence}

As described earlier, universes are types which has types as their elements. That is, universes
are types of types. A function $f : A \to \U$, where $A$ is a type and $\U$ is an universe
is called a type family. Notice that each universe should have another universe as its type.
But the rule $\U : \U$ leads to paradoxes similar to set theory as we will see shortly. 
So we have a hierarchy of universes $\U_1 : \U_2 : \U_3 \dots $.\smallskip.

Given $A, B:\U$ we can define the type of equivalences $A\simeq B$ as functions
with \emph{inverses}. But we already have an equivalence $A = B$ by considering types as points
in the universe. We can also have a map $(A = B) \to (A \simeq B)$. The univalence axiom 
due to Voevodsky says that this itself is an equivalence, i.e we have
\[ (A = B) \simeq (A \simeq B).\]
We will try to have a better understanding of this axiom under the topological interpretation
in section \ref{S:UU}.

\section{Burali-Forti Paradox}\label{S:BFP}

Now we will see an informal description of the Burali-Forti paradox, which demonstrates a
contradiction arising from the existence of a single universe containing 
\emph{everything}.\smallskip

Let $Ord$ be the set of all ordered pairs of the form $(A, <_A)$, where $A$ is a set and
and $<_A$ is an irreflexive, well ordering on $A$. Also $(A, <_A)$ and $(B, <_B)$ are 
considered equal/isomorphic (written as $(A,<_A) \cong (B,<_B)$) if there is an order 
preserving bijection between them. That is, $Ord$ is the universe of all ordinals equal up to 
order preserving bijections. Natural numbers $\mathbb{N}$ with the usual order is an ordinal.
$\mathbb{N}\cup \{\infty \}$ is another example of an ordinal with $n < \infty$ for all
$n \in \mathbb{N}$.\smallskip

Given $a\in A$, let $A_a = \{b \in A,\ b <_A a \}$. $A_a$ inherits an order (say $<_a$)
from $(A, <_A)$. We call $(A_a, <_a)$ to be the initial segment of $a$ in $(A, <_A)$. Let
$I_A = \{(A_a, <_a),\ a\in A\}$ be the set of initial segments. Then 
$(I_A,<_I) \cong (A, <_A)$, where ${ (B,<_B) <_I (C,<_C) }$ if there is an order preserving 
strict inclusion (i.e, not surjective) from $(B,<_B)$ into $(C,<_C)$. Clearly 
$(A_a,<_a) <_I (A_b,<_b)$ iff $a <_A b$.\smallskip

Now define an order $<$ on $Ord$ as, $(B,<_B) < (A, <_A)$ if $(B, <_B) \cong (A_a, <_a)$ 
for some $a \in A$ (equivalently if there is an order preserving strict inclusion 
from $(B,<_B)$ into $(A,<_A)$). Then $(Ord, <)$ becomes a well-ordered set. 
So $(Ord, <) \in Ord$. Notice that this order is defined 
on the equivalence classes on well-ordered sets upto order preserving bijection. Also it can
be proven to be irreflexive.\smallskip

Now take any $(A, <_A)\in Ord$. Notice that $(I_A, <_I)$ is inherited from ${(Ord, <)}$. So
$(A, <_A) < (Ord, <)$, as $(A,<_A) \cong (I_A, <_I)$.\smallskip

Let $(\mathcal{O}, <')$ be the ordinal where, $\mathcal{O} = Ord\sqcup \{\star\}$, for 
$a,b\in Ord$, $a<'b$ if $a<b$, and $a<'\star$ for all $a\in Ord$. Then 
$(Ord,<)< (\mathcal{O},<')$. But $(\mathcal{O},<')<(Ord,<)$ as proved in the last paragraph.
So this gives a contradiction as $<$ is transitive but irreflexive. 

\section{Formulating the Paradox in Type Theory}\label{S:Form in TT}

Now we will see an informal formulation of the Burali-Forti paradox in type theory. 
Difficulties arise in formulating the paradox in type theory because:
\begin{enumerate}
\item Unlike sets types have a higher structure.
\item Under the proposition as types interpretation, proofs of a proposition are not unique.
\end{enumerate}

For example, if we define the type of semigroups as

\[ {Semigroup}:\equiv 	\sum_{A : \U} \sum_{m : A \to A \to A} 
    \prod_{x,y,z:A} m(x,m(y,z)) = m(m(x,y),z) \]
    
i.e., a type $A$, along with a binary operation $m$, and a proof of associativity of $m$; then
by a semigroup (under the topological interpretation) we mean a space $A$ along with a
continuous function $m : A\times A \to A$, and a continuous family of paths from 
$m(x,m(y,z)) = m(m(x,y),z)$. Hence clearly, proofs of a proposition need not be unique. So
we will define the property of being a \textbf{\textit{mere proposition}} as,
${ {isProp}(A) :\equiv \prod_{x,y : A} (x = y) }$. This essentially captures the idea
that proposition are either true (i.e., a proof exists, or the type is inhabited) or false 
(i.e., the type is uninhabited). We define the property of being a \textbf{\textit{set}} as, 
${ {isSet} :\equiv \prod_{x,y:A} {isProp}(x=y)  }$. This captures the idea that sets
have no higher structure, which is all we need to formulate the paradox and also most of
modern mathematics which uses set theory as a model. Because these are not first-class 
objects in type theory a lot of work in formulating the paradox will be proving that certain
objects are sets or mere propositions. We will use $Prop$ to denote the type of 
mere propositions.\smallskip

Another source of difficulty is the absence of excluded middle in type theory. To bypass that
we will need the following definition. 

\begin{definition}\label{D:Acc}
Given a set $A$, and a binary relation $(-<-) : A \to A \to Prop$. Now define being 
accessible by $<$ (i.e., ${Acc}_{<} : A \to \U$) as: If $b$ is accessible for every $b<a$ 
then $a$ is accessible. That is, it has one constructor 
\[ {acc}_< : \prod_{a : A} \left( \prod_{b : A} (b<a) \to {Acc}(b) \right) 
\to {Acc}(a). \]

\end{definition}
We will use accessibility in most of the proofs. We can also prove that accessibility is a 
mere proposition. 

\begin{definition}\label{D:WF}
We will say a binary relation $<$ on $A$ is \textbf{\textit{well-founded}} if every element
is accessible (i.e., we have $f : \prod_{a : A} \to Acc(a)$. Well-foundedness is a mere 
proposition as ${Acc}$ is so. Notice that if $<$ on $A$ is well-founded then for some 
$P : A \to \U$ proving $\prod_{a:A} Acc(a) \to P(a)$ is enough to prove $\prod_{a : A} P(a)$.
\end{definition}

\begin{definition}\label{D:Ext}
We will call $<$ \textbf{\textit{extensional}}, if it is well-founded and we have,
\[ \left( \prod_{c : A}  (c<a) \iff (c<b) \right) \to (a = b),\]
where for two mere propositions $A$ and $B$, we write $A\iff B$ to say that there are 
$f : A \to B$ and $g : B \to A$ (using the definition of ${isProp}$ it can be proven 
that $f$ is an equivalence). Notice that extensionality is same as saying that elements 
with equal initial segments are equal.
\end{definition} 

Extensionality is a mere proposition as given $a,b:A$, $a<b$ is a mere proposition, and so
is well-foundedness. Using univalence we can prove that the type of extensional well-founded
relations is a set.

\begin{definition}\label{D:sim}
If $(A,<)$ and $(B,<)$ are well-founded and extensional, a \textbf{\textit{simulation}} is
a function $f : A \to B$ is such that:
\begin{enumerate}
\item if $a < a'$, then $f(a) < f(a')$, and
\item for all $a : A$ and $b : B$, if $b < f(a)$, then there merely exists an $a' < a$ with
      $f(a') = b$.
\end{enumerate}
\end{definition} 

A simulation is essentially a function from one ordinal (defined next) to another, 
respecting the order. Using well-founded induction we can prove that each simulation is 
injective. Because simulations are functions from one set to another and $a<b$ is a mere 
proposition, injectivity implies that being a simulation is also a mere proposition. Using 
induction on $<$ we can prove, for well-founded and extensional $(A,<)$ and $(B,<)$
there is at most one simulation $f:A \to B$. If we write $A \leq B$ to mean there exists a 
simulation $A \to B$, then $A \leq B$ is a mere proposition. Moreover it is a partial order, 
i.e, $A \leq B$ and $B \leq A$ implies $A = B$. 

\begin{definition}\label{D:Ordinal}
An \textbf{\textit{ordinal}} is a set $A$ with an extensional well-founded relation $<$, which
is also transitive. Given an universe $\U_i$ the type of ordinals in $\U_{i}$, ${Ord}_{\U}$
can be defined as an element in the next universe $\U_{i+1}$.
\end{definition}

Because $a < b$, well-foundedness, and extensionality are mere propositions, so are
transitivity and the property of being an ordinal.\smallskip

Given an ordinal $A$ and $a : A$ let $A_a :\equiv \{ b : A\ |\ b < a\}$ denote the 
\textbf{\textit{initial segment}} of $a$ in $A$. Clearly $A_a$ is also an ordinal. If 
$A_a = A_b$ as ordinals, then the isomorphism should respect the order in $A$. Hence by 
extensionality $a = b$. So the function $a \to A_a$ is injective. 

\begin{definition}\label{D:Bdd sim}
Given two ordinals $A$ and $B$, a simulation $f : A \to B$ is said to be 
\textbf{\textit{bounded}} if there exists $b : B$ such that $A = B_b$. 
\end{definition}

Note that from the discussion above boundedness is a mere proposition. We will write 
$A < B$ to denote that there exists a bounded simulation from $A$ to $B$. Because being a 
simulation is a mere property, so is $<$. Using accessibility we can prove that any 
well-founded relation is irreflexive. Using well-founded induction we can prove that $<$
is a well-founded extensional relation on ${Ord}_{\U}$. Transitivity follows from 
the definition of a simulation. So we have the following theorem.
\begin{theorem}
$({Ord}_{\U_i},<) : \U_{i+1}$ is an ordinal.
\end{theorem}

Given an ordinal $A$ the map $a \mapsto A_a$ is a simulation $A \to {Ord}$ as it is 
injective and respects the orders. Using univalence we can prove that $A = {Ord}_A$. 
If there is a single universe $\U$ containing everything then 
$({Ord}_{\U},<) : {Ord}_{\U}$. So $A < {Ord}_{\U}$ for every ordinal 
$A : {Ord}_{\U}$. But that contradicts irreflexivity of $<$.\smallskip

With minor changes this method can be used to prove a paradox assuming existence of only a
finite number of universes. 

\section{Understanding Univalence}\label{S:UU}

\subsection{Type Families as Fibrations}
Any type family $P : A \to \U$ can be considered as a fibration 
${ {pr}_1 : \sum_{a : A} P(a) }$, where ${pr}_1$ is projection onto the first
co-ordinate. This fibration satisfies \emph{homotopy lifting property} automatically. 
In this interpretation functions $f : \prod_{a : A} P(a)$ become sections. Also a path 
$p : a =_A b$ induces a fibre transport $p_* : P(a) \to P(b)$. We can even prove that the 
fibres are equivalent (with a proper notion of equivalence). So we automatically have 
equivalences arising from equalities. Univalence essentially says that in case of universes
equalities and equivalences are in one to one correspondence with each other in a 
\emph{continuous} manner. Similar results are observed for vector bundles also.

\subsection{Vector Bundles}
A vector bundle is a bundle $p : E \to B$ where inverse image of each element has a 
vector space structure and the bundle is locally trivial. A non-trivial example of a 
vector bundle is the canonical vector bundle $\gamma^n_k$ on the Grassmanian 
$G_k(\mathbb{R}^n)$. This is a 
subbundle of $(G_k(\mathbb{R}^n) \times \mathbb{R}_n,p,G_k(\mathbb{R}^n))$ with the total
space consisting of pairs $(V,x)$ with $x\in V$. $\gamma^n_k$ can be naturally included 
inside $\gamma^{n+1}_k$. In fact we can carry on this way to define $\gamma^{\infty}_k$. 

\subsection{Universes as Universal Bundles}

Given a natural number $k$, the vector bundle $\gamma^{\infty }_n$ serves as a 
universal object/classifying space due to the following results 
($F$ is either $\mathbb{R}$ or $\mathbb{C}$).

\begin{theorem}\label{T:Ind}
Every vector bundle $\mathscr{E}^k$ over a paracompact space $B$ is $B$-isomorphic
to the induced bundle $f^*(\gamma^{\infty}_k)$ for some ${f : B \to G_k (F^{ \infty })}$.
\end{theorem}

\begin{theorem}\label{T:Hom to Eq}
Let $f, g: B \to B'$ be two homotopic maps, where $B$ is a para-compact space, 
and let $\mathscr{E}$ be a vector bundle over $B'$. Then $f^*(\mathscr{E})$ and 
$g^*(\mathscr{E})$ are $B$-isomorphic.
\end{theorem}

\begin{theorem}\label{T:Eq to Hom}
Let $f, g: B \to G_k(F^n)$ be s.t. $f^*(\gamma_k^n)$ and $g^*(\gamma_k^n)$
are $B$-isomorphic and let ${ j: G_k(F^n) \to G_k(F^{2n}) }$ be the natural inclusion. Then the
maps $jf$ and $jg$ are homotopic for $1 \leq n \leq \infty$. 
\end{theorem}

Theorem \ref{T:Ind} is counterpart of the fact that type families are fibrations. Theorem
\ref{T:Hom to Eq} essentially says that homotopies induce equivalences. Similar to type theory
this holds in general for vector bundles modulo some technicalities. Theorem \ref{T:Eq to Hom}
for the case $n = \infty$ proves that $\gamma^{\infty}_k$ is the classifying space in the 
sense that homotopies between maps into the spaces are induced by bundle isomorphisms.
So if we consider $A =_{\U} B$ as the type of homotopies and $A \simeq B$ as the type of
equivalences/isomorphisms, universes can be interpreted as universal bundles where homotopies
and isomorphism are in one-to-one correspondence with each other.
 


\begin{thebibliography}{99}

\bibitem{Hott}
Homotopy Type Theory: Univalent Foundations of Mathematics.

\bibitem{Manin}
A Course in Mathematical Logic for Mathematicians: Yu. I. Manin. 

\bibitem{Husemoller}
Fibre Bundles: Dale Husemoller

\end{thebibliography}

\end{document}