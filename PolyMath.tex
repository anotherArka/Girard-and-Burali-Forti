\documentclass[aspectratio=1610,17pt, ucs]{beamer}

% \usepackage[utf8]{inputenc}
% \usepackage[T1]{fontenc}
% \usepackage[USenglish]{babel}
% \usepackage{graphicx} % graphics
% % \usepackage{mathabx}
% \usepackage{mathpazo}
% % \usepackage{eulervm}
%
% \usepackage[T1]{fontenc}
% \usepackage{lmodern}
%
% \usepackage{minted}
%
% \usepackage{hyperref}
%

\usepackage{listings}
\usetheme{unipassau}

\lstset{
language=scala,
otherkeywords={=, ==, =:=, +, |+|, &&},
basicstyle=\tiny,
literate={~} {$\sim$}{1}
}

\setbeamercovered{invisible}
\setbeamercolor{alerted text}{fg=blue}

\newcommand{\Z}{\mathbb{Z}}
\newcommand{\R}{\mathbb{R}}

\newcommand{\N}{\mathbb{N}}
\newcommand{\Q}{\mathbb{Q}}
\newcommand{\F}{\langle \alpha, \beta \rangle}
\newcommand{\ba}{\bar{\alpha}}
\newcommand{\bb}{\bar{\beta}}
\newcommand{\abab}{\alpha\beta\alpha^{-1}\beta^{-1}}


\newcommand{\co}{\colon\thinspace}

\newtheorem{proposition}{Proposition}
\newtheorem*{quotthm}{Theorem}
\newtheorem{question}{Question}
\newtheorem{conjecture}{Conjecture}



\begin{document}

% \newtheorem{theorem}{Theorem}[section]
% \newtheorem{lemma}[theorem]{Lemma}
% \newtheorem{corollary}[theorem]{Corollary}

% title slide definition
\title{Homogeneous length functions on Groups}
\subtitle{A PolyMath adventure}
\author[Siddhartha Gadgil]{
Siddhartha Gadgil
}
\institute
{
  Department of Mathematics,\\
  Indian Institute of Science.
}


\date{January 31, 2019}
\maketitle

%--------------------------------------------------------------------
%                            Titlepage
%--------------------------------------------------------------------



% \begin{frame}[plain]
%   \titlepage
% \end{frame}

\begin{frame}
  \begin{itemize}[<+->]
    \item On Saturday, December 16, 2017, Terrence Tao posted on his blog a question, from Apoorva Khare.

\begin{question}
    Is there a homogeneous, (conjugacy invariant) length function on the free group on two generators?
\end{question}

    \item Six days later, this was answered in a collaboration involving several mathematicians (and a computer).
    \item This the story of the answer and its discovery.
  \end{itemize}

\end{frame}

\begin{frame}{PolyMath 14 Participants}
  \begin{itemize}
    \item Tobias Fritz, MPI MIS
    \item Siddhartha Gadgil, IISc, Bangalore
    \item Apoorva Khare, IISc, Bangalore
    \item	Pace Nielsen, BYU
    \item Lior Silberman, UBC
    \item Terence Tao, UCLA
  \end{itemize}

\end{frame}

\begin{frame}{Outline}
  \pause
  \tableofcontents[pausesections]
\end{frame}



% -------------------------------------------------------------------
%                            Content
% -------------------------------------------------------------------

\section{The Question}


\begin{frame}{Groups}
  \begin{itemize}[<+->]
    \item A Group $G$ is a set together with 
    \begin{itemize}
      \item an \alert{associative} binary operation $G \times G \to G$,
      \item an \alert{identity} $e$ such that $g \cdot e = e \cdot g = g$ for all $g \in G$,
      \item an \alert{inverse} function $g \mapsto g^{-1}$ such that $g \cdot g^{-1} = g^{-1} \cdot g = e$ for all $g\in G$.
    \end{itemize}
    \item Integers $\Z$ with the addition operation form a group.
    \item Pairs of real numbers with componentwise addition form the group $\R^2$.
    \item For $n\geq 1$, $n \times n$ real matrices with determinant $1$ form a group (called $Sl(n, \R)$).
  \end{itemize}

\end{frame}

\begin{frame}{Length functions}

  \begin{itemize}[<+->]
    \item A \alert{pseudo-length function} on a group $G$ is a function $l: G \to [0, \infty)$ such that
    \begin{itemize}
      \item $l(e) = 0$, where $e\in G$ is the identity,
      \item $l(g^{-1}) = l(g)$ for all $g \in G$ (\textbf{symmetry}),
      \item $l(gh) \leq l(g) + l(h)$ for all $g,h\in G$ (the \textbf{triangle inequality}).
    \end{itemize}
    \item A pseudo-length function $l$ on a group $G$ is said to be a \alert{length function} if $l(g) > 0$ for all $g\in G \setminus \{ e \}$.
    \item Norms on vector spaces, such as $l(x, y) = \sqrt{x^2 + y^2}$ on $\R^2$, are length functions. 
  \end{itemize}

\end{frame}

\begin{frame}{Homogeneity and Conjugacy invariance}
  \begin{itemize}[<+->]
    \item A pseudo-length function $l$ on a group $G$ is said to be \alert{homogeneous} if $l(g^n) = nl(g)$ for all $g\in G$, $n \in\Z$.
    \item Norms are homogeneous -- indeed Apoorva's question was motivated by generalizing \emph{stochastic inequalities}  from Vector spaces with norms.
    \item A pseudo-length function $l$ on a group $G$ is said to be \alert{conjugacy invariant} if $l(ghg^{-1}) = l(h)$ for all $g, h\in G$.
    \item If $G$ is \alert{abelian} ($gh=hg$ for all $g, h\in G$) this holds.
  \end{itemize}

\end{frame}

\begin{frame}{Lengths and Metrics}
  \begin{itemize}[<+->]
    \item Given a length $l: G \to \R$ on a group $G$, we can define a \alert{metric}  on $G$ by $d(x, y) = l(x^{-1}y)$.
    \item This is \alert{left-invariant}, i.e., $d(gx, gy) = d(x, y)$ for all $g, x, y \in G$.
    \item Conversely any left invariant metric gives a length $l(g) := d(e, g)$, with $d(x, y) = l(x^{-1}y)$.
    \item The metric $d$ associated to $l$ is \alert{right-invariant}, (i.e., $d(xg, yg) = d(x, y)$ for all $g, x, y \in G$) if and only if $l$ is \alert{conjugacy invariant}.
  \end{itemize}
\end{frame}

\begin{frame}{The Free Group $\F$}
\begin{itemize}[<+->]
  \item Consider words in $S =\{\alpha$, $\beta$, $\alpha^{-1}$, $\beta^{-1}\}$, where we think of $\alpha^{-1}$ and $\beta^{-1}$ as simply formal symbols.
  \item We regard two words as equal if they are related by a  sequence of moves given by cancellation of pairs of \alert{adjacent} letters that are \alert{inverses}
  of each other.
  \item For example, $\alpha\beta\beta^{-1}\alpha\beta\alpha^{-1} = \alpha\alpha\beta\alpha^{-1}$.
  \item Formally, we define an equivalence relation and consider the corresponding quotient.
  
\end{itemize}
\end{frame}

\begin{frame}{The Free group $\F$}
  \begin{itemize}[<+->]
    \item The group $\F$ is the set of words in $S$ up to the equivalence given above.
    \item Multiplication in $\F$ is given by concatenation, i.e.
  $$(\xi_1\xi_2\dots \xi_n) \cdot (l'_1l'_2\dots l'_m) = \xi_1\xi_2\dots \xi_nl'_1l'_2\dots l'_m$$
    \item The identity $e$ is the empty word.
    \item The inverse of an element is obtained by inverting letters and reversing the order, i.e., $(\xi_1\xi_2\dots \xi_n)^{-1}=\xi_n^{-1}\dots \xi_2^{-1}\xi_1^{-1}$.
  \end{itemize}
  
\end{frame}

\begin{frame}{The Question}
  \pause
  \begin{question}[Apoorva Khare via Terence Tao]
    Is there a function $l: \langle \alpha, \beta\rangle \to [0, \infty)$ on the free group on two generators such that \pause
    \begin{itemize}[<+->]
      \item $l(g) = 0$ if and \alert{only if} $g = e$ (\alert{positivity}).
      \item $l(g^{-1}) = l(g)$ for all $g\in \langle \alpha, \beta\rangle$.
      \item $l(gh) \leq l(g)+ l(h)$ for all $g, h\in \langle \alpha, \beta\rangle$.
      \item $l(ghg^{-1}) = l(h)$ for all $g, h\in \langle \alpha, \beta\rangle$.
      \item $l(g^n) = nl(g)$ for all $g\in \langle \alpha, \beta\rangle$, $n\in\mathbb{Z}$.
    \end{itemize}
\end{question}

\end{frame}

\section{The Quest}

\begin{frame}{Some observations}
  \begin{itemize}[<+->]
    \item By counting the number of occurences of $\alpha$ and $\beta$ with sign, we get a homomorphism $\varphi : \F \to \Z^2$.
    \item The length $l_{\Z^2}(x, y) = |x| + |y|$ on $\Z^2$ 
    induces a homogeneous, conjugacy-invariant pseudo-length 
    $\bar{l}(g) = l_{\Z^2}(\varphi(g))$ on $\F$;
    however,  as $\varphi(\abab) = (0, 0)$, $\bar{l}(\alpha\beta\alpha^{-1}\beta^{-1}) = 0$.
    \item (Fritz) Homogeneity implies conjugacy invariant.
    \item (Tao, Khare) Homogeneity follows from $l(g^2) \geq 2l(g)$ for all $g\in \F$.

  \end{itemize}

\end{frame}


\begin{frame}{The Quest}
  \begin{itemize}
    \item Over the first 4-5 days after the question was posted,
    \begin{itemize}
      \item there were many (failed, but instructive) attempts to construct such length functions;
      \item\pause in particular I focussed on a construction using \alert{non-crossing matchings}, \pause but this failed homogeneity;
      \item\pause the failures of various constructions led to the general feeling that $l(\alpha\beta\alpha^{-1}\beta^{-1}) = 0$;
      \item\pause increasingly sharp bounds and methods of combining bounds were found, but there was no visible path to proving $l(\alpha\beta\alpha^{-1}\beta^{-1}) = 0$.
    \end{itemize}
    \item\pause On Thursday morning I posted a proof of a computer-assisted bound.
  \end{itemize}

\end{frame}

\begin{frame}{Proof which I posted online}
  \pause
  {\small Proof of a bound on $l(\alpha\beta\alpha^{-1}\beta^{-1})$ for $l$ a homogeneous, conjugacy invariant length function with $l(\alpha), l(\beta) \leq 1$.}\pause
  \tiny
  \begin{itemize}
    \item $|\bar{a}| \leq 1.0$
    \item $|\bar{b}\bar{a}b| \leq 1.0$ using $|\bar{a}| \leq 1.0$
    \item $|\bar{b}| \leq 1.0$
    \item $|a\bar{b}\bar{a}| \leq 1.0$ using $|\bar{b}| \leq 1.0$
    \item $|\bar{a}\bar{b}ab\bar{a}\bar{b}| \leq 2.0$ using $|\bar{a}\bar{b}a| \leq 1.0$ and $|b\bar{a}\bar{b}| \leq 1.0$
    \item ... (119 lines)
    \item $|ab\bar{a}\bar{b}ab\bar{a}\bar{b}ab\bar{a}\bar{b}ab\bar{a}\bar{b}ab\bar{a}\bar{b}ab\bar{a}\bar{b}ab\bar{a}\bar{b}ab\bar{a}\bar{b}ab\bar{a}\bar{b}ab\bar{a}\bar{b}ab\bar{a}\bar{b}ab\bar{a}\bar{b}ab\bar{a}\bar{b}ab\bar{a}\bar{b}ab\bar{a}\bar{b}ab\bar{a}\bar{b}ab\bar{a}\bar{b}| \leq 13.859649122807017$ using $|ab\bar{a}| \leq 1.0$ and $|\bar{b}ab\bar{a}\bar{b}ab\bar{a}\bar{b}ab\bar{a}\bar{b}ab\bar{a}\bar{b}ab\bar{a}\bar{b}ab\bar{a}\bar{b}ab\bar{a}\bar{b}ab\bar{a}\bar{b}ab\bar{a}\bar{b}ab\bar{a}\bar{b}ab\bar{a}\bar{b}ab\bar{a}\bar{b}ab\bar{a}\bar{b}ab\bar{a}\bar{b}ab\bar{a}\bar{b}ab\bar{a}\bar{b}| \leq 12.859649122807017$
    \item $|ab\bar{a}\bar{b}| \leq 0.8152734778121775$ using $|ab\bar{a}\bar{b}ab\bar{a}\bar{b}ab\bar{a}\bar{b}ab\bar{a}\bar{b}ab\bar{a}\bar{b}ab\bar{a}\bar{b}ab\bar{a}\bar{b}ab\bar{a}\bar{b}ab\bar{a}\bar{b}ab\bar{a}\bar{b}ab\bar{a}\bar{b}ab\bar{a}\bar{b}ab\bar{a}\bar{b}ab\bar{a}\bar{b}ab\bar{a}\bar{b}ab\bar{a}\bar{b}ab\bar{a}\bar{b}| \leq 13.859649122807017$ by taking 17th power.
  \end{itemize}
  \pause
  {\small i.e., $l(\alpha, \beta) \leq 0.8152734778121775$}
\end{frame}

\begin{frame}
  \begin{itemize}[<+->]
    \item The computer-generated proof was studied by Pace Nielsen, who extracted the \alert{internal repetition} trick.
    \item This was extended by Pace Nielsen and Tobias Fritz and generalized by Terence Tao.
    \item From this Fritz obtained the key lemma:
    \begin{lemma}
      Let $f(m,k)=l(x^m (xyx^{-1}y^{-1})^k)$. Then $$f(m,k)\leq \frac{f(m-1,k)+f(m+1,k-1)}{2}.$$
    \end{lemma}
    \vspace{-1cm}
    \item Using Probability, Tao showed $l(\alpha\beta\alpha^{-1}\beta^{-1}) = 0$.
  \end{itemize}

\end{frame}



\section{Computer Bounds and Proofs}

\begin{frame}{Bounds from Conjugacy invariance}
  \begin{itemize}[<+->]
    \item Fix a conjugacy-invariant, \alert{normalized} length function $l: \F \to \R$, i.e. with $l(\alpha), l(\beta)\leq 1$.
    \item Let $g=\xi_1\xi_2\dots \xi_n$ with $n \geq 1$.
    \begin{itemize}
      \item\label{plus1} By the triangle inequality $$l(g) \leq 1 + l(\xi_2\xi_3\dots \xi_n).$$
      \item\label{pair} If $\xi_k = \xi_1^{-1}$, by the triangle inequality and conjugacy invariance
       $$l(g)\leq l(\xi_2\xi_3\dots \xi_{k-1}) + l(\xi_{k+1}\xi_{k+2}\dots \xi_n)$$ as 
         $l(\xi_1\xi_2\dots \xi_k) = l(\xi_1\xi_2\dots \xi_{k-1}\xi_1^{-1}) = l(\xi_2\xi_2\dots \xi_{k-1})$.
      
    \end{itemize}
  \end{itemize}
  
\end{frame}

\begin{frame}{The recursive algorithm}
  For $g\in F$, compute $L(g)$ such that $l(g)\leq L(g)$ by:
  \begin{itemize}[<+->]
    \item If $g = e$ is the empty word, \alert{define} $L(g) := 0$.
    \item If $g=\xi_1$ has exactly one letter, \alert{define} $L(g) := 1$.
    \item If $g = \xi_1\xi_2\dots \xi_n$ has at least two letters:
    \begin{itemize}
      \item let $\lambda_0 = 1 + L(\xi_2\xi_3\dots \xi_n)$ (computed recursively).
      \item let $\Lambda$ be the (possibly empty) set
      $$\{L(\xi_2\xi_3\dots \xi_{k-1}) + L(\xi_{k+1}\xi_{k+2}\dots \xi_n): 2 \leq k\leq n, \xi_k = \xi_1^{-1}\}$$
      \item \alert{define} $L(g) := \min(\{\lambda_0\}\cup \Lambda)$.
    \end{itemize}
  \end{itemize}
\end{frame}

\begin{frame}{Ad hoc bounds using Homogeneity}
  \begin{itemize}[<+->]
    \item For chosen $g\in \F$, $n \geq 1$, homogeneity gives $l(g)\leq L(g^n)/n$ for $l$ a normalized, homogeneous length function on $\F$.
    \item Further, we can use this (in general improved) bound (in place of $L(g)$) recursively in the above algorithm.
    \item We computed such bounds in interactive sessions. 
    \item The words used were $\alpha(\alpha\beta\alpha^{-1}\beta^{-1})^k$, chosen based on non-homogeneity of the conjugacy-invariant length function $l_{WC}$ based on non-crossing matchings.
  \end{itemize}
  
\end{frame}

\begin{frame}{From bounds to Proofs}
  \begin{itemize}
    \item\pause Rather than (recursively) generating just bounds, we can recursively generate \alert{proofs} of bounds.
    \item\pause These were in terms of \alert{domain specific foundations}, which could be viewed as embedded in Homotopy Type Theory; \pause which is a system of foundations of mathematics related to topology.
    \item\pause In this case, we can instead view our algorithm as just keeping track of relevant inequalities.
  \end{itemize}
  
\end{frame}

\begin{frame}[fragile]{Domain specific foundations in scala}
 
  {\small 
  \begin{itemize}
    \item Proofs were represented as \alert{objects} of a specific \alert{type}.
    \item The \alert{correctness} was independent of \alert{discovery}. 
  \end{itemize}
  }
  \pause
  \begin{lstlisting}
sealed abstract class LinNormBound(val word: Word, val bound: Double)

final case class Gen(n: Int) extends LinNormBound(Word(Vector(n)), 1)

final case class ConjGen(n: Int,pf: LinNormBound) extends
      LinNormBound(n +: pf.word :+ (-n), pf.bound)

final case class Triang(
  pf1: LinNormBound, pf2: LinNormBound) extends
      LinNormBound( pf1.word ++ pf2.word, pf1.bound + pf2.bound)

final case class PowerBound(
  baseword: Word, n: Int, pf: LinNormBound) extends
    LinNormBound(baseword, pf.bound/n){require(pf.word == baseword.pow(n))}

final case object Empty extends LinNormBound(Word(Vector()), 0)
  \end{lstlisting}

\end{frame}

\section{The Theorem and Proof}

\begin{frame}{The main results}
  \begin{theorem}
    For any group $G$, every homogeneous pseudo-length $l: G \to \R$ is the pullback of a homogeneous pseudo-length on the abelianization $G/ [G, G]$. 
  \end{theorem}

  \pause 

  \begin{corollary}
    If $G$ is not abelian (e.g. $G = \mathbb{F}_2$) there is no homogeneous length function on $G$.
  \end{corollary}

\end{frame}

\begin{frame}{Internal Repetition trick}
  \begin{lemma}
    If $x = s(wy)s^{-1} = t(zw^{-1})t^{-1}$, we have $l(x)\leq \frac{l(y)+ l(z)}{2}$.
  \end{lemma}
  \pause
      \begin{itemize}[<+->]
        \item $\begin{aligned}
          l(x^nx^n) &= l(s(wy)^ns^{-1}t(zw^{-1})^nt^{-1}) \\ 
          &\leq n(l(y) +l(z)) + 2(l(s) + l(t))
        \end{aligned}$
      \begin{center}      
        \frame{Figure purged}
      \end{center}
      \item Use $l(x) = \frac{l(x^nx^n)}{2n}$ and take limits.
      \end{itemize}
    
\end{frame}

\begin{frame}{Tao's probability theory argument}
  \begin{itemize}[<+->]
    
    \item The inequality $f(m,k)\leq \frac{f(m-1,k)+f(m+1,k-1)}{2}$ can be interpreted as the average value of $f$ non-decreasing along the random walk on $\Z^2$ where we move by $(-1, 0)$ or $(1, -1)$ with equal probability. 
    \item The average movement in a step was $(0, -1/2)$.
    \item Hence taking $2n$ steps starting at $(0, n)$ gives an upper bound for $f(0, 2n)=l((\alpha\beta\alpha^{-1}\beta^{-1})^n)$ by the average length for a distribution centered at the origin. 
    \item This was bounded using the Chebyshev inequality.
    
    
  \end{itemize}
\end{frame}

\section{Epilogue}

\begin{frame}{On the computer proof}
  \begin{itemize}
    \item A limitation was that the elements for which we applied homogeneity were selected by hand.
    \item \pause More importantly, in our representations of proofs, the bounds were only for concrete group elements.
    \item\pause In particular, we could not 
    \begin{itemize}
      \item\pause represent inequalities for expressions,
      \item\pause use induction.
    \end{itemize}
    \item\pause Would want proof in complete foundations;\pause\ which I completed a few days after the PolyMath proof (in my own implementation of HoTT).
  \end{itemize}
\end{frame}

\begin{frame}{Quasification}
  \begin{itemize}[<+->]
    \item The function $l: G \to [0, \infty)$ is a \alert{quasi-pseudo-length function}  if there exists $c \in \mathbb{R}$ such that
    $l(gh) \leq l(g) + l(h) + c$, for all $g,h\in G$.
    \item We see that for a homogeneous quasi-pseudo-length function, $l(xyx^{-1}y^{-1}) \leq 4c$ for all $x, y\in G$.
    % \item Free groups have homogeneous quasi-pseudo-length functions that are not pullbacks of norms.
    \item For a group with vanishing \alert{stable commutator length}, e.g. $G=Sl(3 , \mathbb{Z})$, any homogeneous
    quasi-pseudo-length function is equivalent to a pullback from $G/[G, G]$. 
  \end{itemize}
  
\end{frame}

\begin{frame}{Afterword}
  \begin{itemize}
    \item This work became \alert{PolyMath 14}, and has been published in \emph{Algebra \& Number Theory}.
    \item\pause The work was a spontaneous collaboration across (at least) three continents, and a range of skills.
    \item\pause A \alert{computer generated} but \alert{human readable} proof was read, understood, generalized and abstracted by mathematicians to obtain the key lemma in an interesting mathematical result;\pause\  this is perhaps the first time this has happened.
  \end{itemize}
\end{frame}



%%%%%%%%%%%%%%%%%%

\end{document}
